\documentclass[a4paper,12pt]{article}

\usepackage[T2A]{fontenc}
\usepackage[utf8]{inputenc}
\usepackage[english,russian]{babel}
\usepackage{fontspec}
\setmainfont[Scale=1.167]{Times New Roman}

\usepackage{unicode-math}

\usepackage{microtype}
\usepackage[linesnumbered,ruled,vlined]{algorithm2e}

\usepackage{amsmath, amsfonts, amsthm}
\usepackage{listings}
\usepackage{enumerate}
\usepackage{float}
\usepackage{graphicx}
\usepackage{nameref}
\usepackage{hyperref}
\usepackage{tabularx}
\usepackage{makecell}
\usepackage{indentfirst}
\usepackage{booktabs}
\usepackage{derivative}
\usepackage{physics}

\usepackage{tikz}
\usepackage{tikzscale}
\usetikzlibrary{decorations.pathreplacing,calligraphy}
\usetikzlibrary{arrows.meta}
\usetikzlibrary{backgrounds,fit,positioning}
\usetikzlibrary{external}

\usepackage[newfloat]{minted}

\setminted{
  frame=lines,
  framesep=2mm,
  baselinestretch=1.2,
  fontsize=\footnotesize,
  linenos,
  breaklines
}

\usepackage[top=3cm,bottom=3cm,left=2.5cm,right=2.5cm]{geometry}
\hypersetup{
	colorlinks=true,
	linkcolor=blue,
	filecolor=magenta,
	urlcolor=cyan
}
\urlstyle{same}

\begin{document}
	\begin{titlepage}	% начало титульной страницы

	\begin{center}		% выравнивание по центру

		\large Санкт-Петербургский политехнический университет Петра Великого\\
		\large Институт компьютерных наук и технологий\\
		\large Высшая школа программной инженерии \\[6cm]
		% название института, затем отступ 6см

    \huge Лабораторная работа № 1\\[0.5cm] % название работы, затем отступ 0,5см
		\large по дисциплине\\[0.1cm]
		\large <<Математические модели>>

	\end{center}

		\noindent\large Выполнил: \hfill \large Ферапонтов М.В.\\
		\noindent\large Группа: \hfill \large гр. 3530904/00104\\

		\noindent\large Проверил: \hfill \large Воскобойников С. П.

	\vfill % заполнить всё доступное ниже пространство

	\begin{center}
	\large Санкт-Петербург\\
	\large \the\year % вывести дату
	\end{center} % закончить выравнивание по центру

\end{titlepage} % конец титульной страницы

\vfill % заполнить всё доступное ниже пространство

	\newpage
	\tableofcontents
	\newpage

	\section{Вступление}
	\subsection{Постановка задачи}

	Вариант CP. Используя интегро-интерполяционный метод (метод баланса), разработать программу для моделирования стационарного распределения температуры в полом цилиндре, описываемого математической моделью вида:
	\begin{align*}
		-\left[ \frac{1}{r} \frac{d}{dr} \left(r k(r)\frac{du}{dr} \right)\ -\ q(r)u \right]
		&= f(r),\ r \in \left[ R_L,\ R_R\right],\ R_L > 0,
		\\
		 0 < c_1 \leq k(r) \leq c_2,\ 0 \leq q(r)&
	\end{align*}
	Граничные условия: \newline
	\begin{align*}
		&k \left. \frac{du}{dr}\right\vert_{r = R_L} = -\nu _1
		&-k \left. \frac{du}{dr}\right\vert_{r = R_R} = -\nu_2
	\end{align*}
	\subsection{Используемое ПО}

	\begin{enumerate}
		\item \href{https://www.boost.org/}{Boost library} - библиотека для тестирования и других функций
		\item \href{https://www.gnu.org/software/gsl/}{\textbf{GSL}} - GNU Scientific Library. Математическая библиотека для C и C++.
	\end{enumerate}
	\newpage

	\section{Основная часть}
	\subsection{Разностная схема}
Введем основную сетку:
\begin{align*}
  &N_r - \text{число разбиений на} [0, R] & &N_z - \text{число разбиений на} [0, L] \\
  &r_0 < r_1 < \cdots < r_N & &z_0 < z_1 < \cdots < z_N \\
  &r_0 = 0,\quad r_N = R & &z_0 = 0,\quad z_N = L \\
  &h_i = \frac{R - 0}{N_r},\quad i=0,\cdots, N_r & &h_j = \frac{L - 0}{N_z},\quad j=0,\cdots, N_z 
\end{align*}

Введем дополнительную сетку:
\begin{align*}
  &r_{i-\frac{1}{2}} = \frac{r_i + r_{i - 1}}{2}\quad i=1,\cdots, N_r & &z_{j-\frac{1}{2}} = \frac{z_j + z_{j - 1}}{2}\quad j=1,\cdots, N_z \\
  &  \hbar_i = \begin{cases}
    \frac{h_i + 1}{2},\ i = 0 \\ \\
    \frac{h_i + h_{i+1}}{2},\ i = 1, 2, \dots, N_r-1 \\ \\
    \frac{h_i}{2},\ i = N_r
  \end{cases} &
  &   \hbar_j = \begin{cases}
    \frac{h_j + 1}{2},\ j = 0 \\ \\
    \frac{h_j + h_{j+1}}{2},\ j = 1, 2, \dots, N_z-1 \\ \\
    \frac{h_j}{2},\ j = N_z
  \end{cases}
\end{align*}

Преобразуем наше начальное уравнение

\[
  - \left [ \pdv{}{r} \left ( r k_1(r, z) \pdv{u}{r} \right ) 
  + \pdv{}{z} \left ( rk_2(r, z)\pdv{u}{v} \right ) \right ] = rf(r, z)
\]

Проинтегрируем уравнение внутри интервала:

\[
  - \mLim{+\frac{1}{2}}{-\frac{1}{2}}{+\frac{1}{2}}{-\frac{1}{2}} \left [ \pdv{}{r} \left ( r k_1(r, z) \pdv{u}{r} \right ) 
  + \pdv{}{z} \left ( rk_2(r, z)\pdv{u}{v} \right ) \right ] dr dz = \mLim{+\frac{1}{2}}{-\frac{1}{2}}{+\frac{1}{2}}{-\frac{1}{2}} rf(r, z) dr dz
\]

Получим:

\begin{align*}
  &- \left [
   \mLimS{z}{+\frac{1}{2}}{-\frac{1}{2}}  \left . r k_1(r, z) \pdv{u}{r} \right \vert_{r = r_{i + \frac{1}{2}}} dz
  - \mLimS{z}{+\frac{1}{2}}{-\frac{1}{2}} \left . r k_1(r, z) \pdv{u}{r} \right \vert_{r = r_{i - \frac{1}{2}}} dz
  \right . \\
  &\left . + \mLimS{r}{+\frac{1}{2}}{-\frac{1}{2}} \left . rk_2(r, z)\pdv{u}{v} \right \vert_{z = z_{j + \frac{1}{2}}} dr
  - \mLimS{r}{+\frac{1}{2}}{-\frac{1}{2}} \left . rk_2(r, z)\pdv{u}{v} \right \vert_{z = z_{j - \frac{1}{2}}} dr
  \right ] = \mLim{+\frac{1}{2}}{-\frac{1}{2}}{+\frac{1}{2}}{-\frac{1}{2}} rf(r, z) dr dz
\end{align*}

Воспользуемся формулами численного дифференцирования:
\[
  \left . k_1(r, z) \pdv{u}{r} \right \vert_{r = r_{i - \frac{1}{2}}}
  \approx k_1(r_{i - \frac{1}{2}}, z) 
  \frac{v_{i, j} - v_{i - 1, j}}{h_i}
\]

\[
  \left . k_2(r, z) \pdv{u}{r} \right \vert_{z = z_{j - \frac{1}{2}}}
  \approx k_2(r, z_{j - \frac{1}{2}}) 
  \frac{v_{i, j} - v_{i, j - 1}}{h_j}
\]

Также воспользуемся формулой средних прямоугольников:
\[
  \mLimS{r}{+\frac{1}{2}}{-\frac{1}{2}} r \varphi(r, z) dr
  = \hbar_i r_i \varphi_i
\]
\[
  \mLim{+\frac{1}{2}}{-\frac{1}{2}}{+\frac{1}{2}}{-\frac{1}{2}} r \varphi(r, z) drdz
  = \hbar_i\hbar_j r_i \varphi_{i, j}
\]

В итоге получаем разностную схему внутри интервала:
\begin{align*}
  - \left [ 
  \hbar_j r_{i+\frac{1}{2}} k_1(r_{i+\frac{1}{2}}, z_j) \frac{v_{i+1, j} - v_{i, j}}{h_{i + 1}}
  - \hbar_j r_{i-\frac{1}{2}} k_1(r_{i-\frac{1}{2}}, z_j) \frac{v_{i, j} - v_{i - 1, j}}{h_{i}}
  \right . \\
  \left .
  + \hbar_i r_{i+\frac{1}{2}} k_2(r_i, z_{j+\frac{1}{2}}) \frac{v_{i, j + 1} - v_{i, j}}{h_{j + 1}}
  - \hbar_i r_{i-\frac{1}{2}} k_2(r_i, z_{j-\frac{1}{2}}) \frac{v_{i, j} - v_{i, j - 1}}{h_j}
  \right ]  = \hbar_i \hbar_j r_i f_{i, j}
\end{align*}

Теперь найдем значение разностной схемы на углах и границах интервалов

\subsubsection{На левой границе}

Проинтегрируем наше уравнение в $ i = 0 $ и $ z $ внутри промежутка
\[
  - \mLim{+\frac{1}{2}}{}{+\frac{1}{2}}{-\frac{1}{2}} \left [ \pdv{}{r} \left ( r k_1(r, z) \pdv{u}{r} \right ) 
  + \pdv{}{z} \left ( rk_2(r, z)\pdv{u}{v} \right ) \right ] dr dz = \mLim{+\frac{1}{2}}{}{+\frac{1}{2}}{-\frac{1}{2}} rf(r, z) dr dz
\]

Получаем:
\begin{align*}
  - \left [
   \mLimS{z}{+\frac{1}{2}}{-\frac{1}{2}}  \left . r k_1(r, z) \pdv{u}{r} \right \vert_{r = r_{i + \frac{1}{2}}} dz
  - \mLimS{z}{+\frac{1}{2}}{-\frac{1}{2}} \left . r k_1(r, z) \pdv{u}{r} \right \vert_{r = r_{i}} dz
  \right . \\
  \left . + \mLimS{r}{+\frac{1}{2}}{} \left . rk_2(r, z)\pdv{u}{v} \right \vert_{z = z_{j + \frac{1}{2}}} dr
  - \mLimS{r}{+\frac{1}{2}}{} \left . rk_2(r, z)\pdv{u}{v} \right \vert_{z = z_{j - \frac{1}{2}}} dr
  \right ] = \mLim{+\frac{1}{2}}{}{+\frac{1}{2}}{-\frac{1}{2}} rf(r, z) dr dz
\end{align*}

Имеем граничное условие:
\[
  \left . u \right \vert_{r=0} - \text{ограничено, т. е } \left . \frac{\partial u}{ \partial r} \right |_{r = 0} = 0
\]

\[
  \mLimS{r}{+ \frac{1}{2}}{} rf dr \approx f_i \mLimS{r}{ + \frac{1}{2}}{} r dr = 
  f_i \frac{r^2_{i + \frac{1}{2}}}{2} = h_r f_i \frac{r_{i + \frac{1}{2}}}{2},
  \quad i = 0, \quad r_i = 0, r_{i + \frac{1}{2}} = \frac{h_r}{2}
\]


Получаем разностную схему:
\begin{align*}
  &- \left [ 
  \hbar_j r_{i+\frac{1}{2}} k_1(r_{i+\frac{1}{2}}, z_j) \frac{v_{i+1, j} - v_{i, j}}{h_{i + 1}}
  - 0
  \right . \\
  &\left .
  + \hbar_i r_{i+\frac{1}{2}} k_2(r_i, z_{j+\frac{1}{2}}) \frac{v_{i, j + 1} - v_{i, j}}{h_{j + 1}}
  - \hbar_i r_{i-\frac{1}{2}} k_2(r_i, z_{j-\frac{1}{2}}) \frac{v_{i, j} - v_{i, j - 1}}{h_j}
  \right ]  = \hbar_i \hbar_j r_i f_{i, j}
\end{align*}

\subsubsection{На правой границе}
Проинтегрируем наше уравнение в $ i = N_x $ и $ z $ внутри промежутка
\[
  - \mLim{}{-\frac{1}{2}}{+\frac{1}{2}}{-\frac{1}{2}} \left [ \pdv{}{r} \left ( r k_1(r, z) \pdv{u}{r} \right ) 
  + \pdv{}{z} \left ( rk_2(r, z)\pdv{u}{v} \right ) \right ]dr dz = \mLim{}{-\frac{1}{2}}{+\frac{1}{2}}{-\frac{1}{2}} rf(r, z)dr dz
\]

Получаем:
\begin{align*}
  &- \left [
   \mLimS{z}{+\frac{1}{2}}{-\frac{1}{2}}  \left . r k_1(r, z) \pdv{u}{r} \right \vert_{r = r_{i}} dz
  - \mLimS{z}{+\frac{1}{2}}{-\frac{1}{2}} \left . r k_1(r, z) \pdv{u}{r} \right \vert_{r = r_{i - \frac{1}{2}}} dz
  \right . \\
  &\left . + \mLimS{r}{}{-\frac{1}{2}} \left . rk_2(r, z)\pdv{u}{v} \right \vert_{z = z_{j + \frac{1}{2}}} dr
  - \mLimS{r}{}{-\frac{1}{2}} \left . rk_2(r, z)\pdv{u}{v} \right \vert_{z = z_{j - \frac{1}{2}}} dr
  \right ] = \mLim{}{-\frac{1}{2}}{+\frac{1}{2}}{-\frac{1}{2}} rf(r, z) dr dz
\end{align*}

Имеем граничное условие:
\[
  \left . -k_1 \pdv{u}{r} \right \vert_{r=R} = \chi_2 \left . u \right \vert_{r=R} - \varphi_2(z)
\]

Получаем разностную схему:
\begin{align*}
  &- \left [ 
  -\hbar_j ( \chi_2 v_i - \varphi_2(z))
  - \hbar_j r_{i-\frac{1}{2}} k_1(r_{i-\frac{1}{2}}, z_j) \frac{v_{i, j} - v_{i - 1, j}}{h_{i}}
  \right . \\
  &\left .
  + \hbar_i r_{i+\frac{1}{2}} k_2(r_i, z_{j+\frac{1}{2}}) \frac{v_{i, j + 1} - v_{i, j}}{h_{j + 1}}
  - \hbar_i r_{i-\frac{1}{2}} k_2(r_i, z_{j-\frac{1}{2}}) \frac{v_{i, j} - v_{i, j - 1}}{h_j}
  \right ]  = \hbar_i \hbar_j r_i f_{i, j}
\end{align*}

\subsubsection{На нижней границе}
Проинтегрируем наше уравнение $ j = 0 $ и $ i $ внутри промежутка
\[
  - \mLim{+\frac{1}{2}}{-\frac{1}{2}}{+\frac{1}{2}}{} \left [ \pdv{}{r} \left ( r k_1(r, z) \pdv{u}{r} \right ) 
  + \pdv{}{z} \left ( rk_2(r, z)\pdv{u}{v} \right ) \right ] drdz = \mLim{+\frac{1}{2}}{-\frac{1}{2}}{+\frac{1}{2}}{} rf(r, z) dr dz
\]

Получаем:
\begin{align*}
  &- \left [
   \mLimS{z}{+\frac{1}{2}}{}  \left . r k_1(r, z) \pdv{u}{r} \right \vert_{r = r_{i + \frac{1}{2}}} dz
  - \mLimS{z}{+\frac{1}{2}}{} \left . r k_1(r, z) \pdv{u}{r} \right \vert_{r = r_{i - \frac{1}{2}}} dz
  \right . \\
  &\left . + \mLimS{r}{+\frac{1}{2}}{-\frac{1}{2}} \left . rk_2(r, z)\pdv{u}{v} \right \vert_{z = z_{j + \frac{1}{2}}} dr
  - \mLimS{r}{+\frac{1}{2}}{-\frac{1}{2}} \left . rk_2(r, z)\pdv{u}{v} \right \vert_{z = z_{j}} dr
  \right ] = \mLim{+\frac{1}{2}}{-\frac{1}{2}}{+\frac{1}{2}}{} rf(r, z) dr dz
\end{align*}

Имеем граничное условие:
\[
  \left . k_2 \pdv{u}{z} \right \vert_{z=0} = \chi_3 \left . u \right \vert_{z=0} - \varphi_3(r)
\]

Получаем разностную схему:
\begin{align*}
  - \left [ 
  \hbar_j r_{i+\frac{1}{2}} k_1(r_{i+\frac{1}{2}}, z_j) \frac{v_{i+1, j} - v_{i, j}}{h_{i + 1}}
  - \hbar_j r_{i-\frac{1}{2}} k_1(r_{i-\frac{1}{2}}, z_j) \frac{v_{i, j} - v_{i - 1, j}}{h_{i}}
  \right . \\
  \left .
  + \hbar_i r_{i+\frac{1}{2}} k_2(r_i, z_{j+\frac{1}{2}}) \frac{v_{i, j + 1} - v_{i, j}}{h_{j + 1}}
  - \hbar_i(\chi_3 v_i - \varphi_3(r))
  \right ]  = \hbar_i \hbar_j r_i f_{i, j}
\end{align*}

\subsubsection{На верхней границе}
Имеем граничное условие:
\[
  \left . u \right \vert_{z=L} = \varphi_r(r) 
\]

\subsubsection{Правый-нижний угол}

Проинегрируем наше уравнение
\[
  - \mLim{}{-\frac{1}{2}}{+\frac{1}{2}}{} \left [ \pdv{}{r} \left ( r k_1(r, z) \pdv{u}{r} \right ) 
  + \pdv{}{z} \left ( rk_2(r, z)\pdv{u}{v} \right ) \right ] = \mLim{}{-\frac{1}{2}}{+\frac{1}{2}}{} rf(r, z)
\]

Получаем:

\begin{align*}
  &- \left [
   \mLimS{z}{+\frac{1}{2}}{}  \left . r k_1(r, z) \pdv{u}{r} \right \vert_{r = r_i} dz
  - \mLimS{z}{+\frac{1}{2}}{} \left . r k_1(r, z) \pdv{u}{r} \right \vert_{r = r_{i - \frac{1}{2}}} dz
  \right . \\
  &\left . + \mLimS{r}{}{-\frac{1}{2}} \left . rk_2(r, z)\pdv{u}{v} \right \vert_{z = z_{j + \frac{1}{2}}} dr
  - \mLimS{r}{}{-\frac{1}{2}} \left . rk_2(r, z)\pdv{u}{v} \right \vert_{z = z_{j}} dr
  \right ] = \mLim{}{-\frac{1}{2}}{+\frac{1}{2}}{} rf(r, z) dr dz
\end{align*}

Имеем граничные условия:

\begin{align*}
  &\left . -k_1 \pdv{u}{r} \right \vert_{r=R} = \chi_2 \left . u \right \vert_{r=R} - \varphi_2(z) \\
  &\left . k_2 \pdv{u}{z} \right \vert_{z=0} = \chi_3 \left . u \right \vert_{z=0} - \varphi_3(r) 
\end{align*}

Получаем разностную схему:

\begin{align*}
  - \left [ 
  -\hbar_j (\chi_2 \left . u \right \vert_{r=R} - \varphi_2(z) )
  - \hbar_j r_{i-\frac{1}{2}} k_1(r_{i-\frac{1}{2}}, z_j) \frac{v_{i, j} - v_{i - 1, j}}{h_{i}}
  \right . \\
  \left .
  + \hbar_i r_{i+\frac{1}{2}} k_2(r_i, z_{j+\frac{1}{2}}) \frac{v_{i, j + 1} - v_{i, j}}{h_{j + 1}}
  - \hbar_i(\chi_3 v_i - \varphi_3(r))
  \right ]  = \hbar_i \hbar_j r_i f_{i, j}
\end{align*}

Другие углы нам известны.

% \subsubsection{Правый-верхний угол}
% \[
%   - \mLim{}{-\frac{1}{2}}{}{-\frac{1}{2}} \left [ \pdv{}{r} \left ( r k_1(r, z) \pdv{u}{r} \right ) 
%   + \pdv{}{z} \left ( rk_2(r, z)\pdv{u}{v} \right ) \right ] = \mLim{}{-\frac{1}{2}}{}{-\frac{1}{2}} rf(r, z)
% \]


	\input{tex/tma.tex}
	\newpage

	\subsection{Оценка погрешности}

\subsubsection{Невязка разностной схемы}

\[
  Av = g,\ A - (N + 1) r (N + 1), v, g \in R^{(N + 1)}
\]

Пусть $ v $ - это точное решение разностной схемы, $ u $ -  точное решение дифференциального уравнения, 
$ \tilde{v} $ - полученное решение разностной схемы.

Ищем погрешность решения разностной схемы:
\[
  \varepsilon = \tilde{v} - u
\]

Введем обозначения:
\begin{itemize}
  \item Погрешность решения системы линейных алгебраических уравнений
  \[ z = \tilde{v} - v \]
  \item Погрешность от аппроксимации дифференциального уравнения разностной схемой
  \[ \zeta = v - u \]
  \item Невязка разностной схемы
  \[ \xi = g - Au \]
  \item Невязка алгебраической системы
  \[ r = g - A\tilde{v} \]
\end{itemize}

\subsubsection{Структура погрешности разностной схемы}
\[
  \left\lVert \varepsilon \right\rVert = \left\lVert \tilde{v} - u \right\rVert =
  \left\lVert \tilde{v} - v + v - u \right\rVert = \left\lVert z + \zeta  \right\rVert \leq \left\lVert z \right\rVert
  + \left\lVert \zeta \right\rVert 
\]

Для $\left\lVert \zeta\right\rVert$:
\[
  \xi = g - Au = A(A^{-1}g - u) = A(v - u)
\]
\[
  A\zeta  = \xi
\]

Тем самым погрешность от аппроксимации дифференциального уравнения разностной схемой, связана с невязкой разностной схемы:
\[
  \zeta = A^{-1} \xi 
\]

Для $\left\lVert z \right\rVert$:
\[
  r = g - A\tilde{v} = A(A^{-1}g - \tilde{v}) = A(v - \tilde{v})
\]
\[
  Az = r
\]
Тем самым погрешность решения системы линейных алгебраических уравнений, связана с невязкой алгебраической системы:
\[
  z = A^{-1}r
\]

Подставим в наше неравенство, тем самым получаем:
\[
  \left\lVert \varepsilon \right\rVert \leq \left\lVert A^{-1}r \right\rVert + \left\lVert A^{-1}\xi \right\rVert
  \leq \left\lVert A^{-1} \right\rVert ( \left\lVert r \right\rVert  + \left\lVert \xi \right\rVert)
  \quad \left\lVert A^{-1}\right\rVert  < C
\]

\subsubsection{Вклад от погрешности решения системы алгебраических уравнений}
\[
  \left\lVert z \right\rVert \leq \left\lVert A^{-1} \right\rVert \left\lVert r \right\rVert =
  \left\lVert A \right\rVert \left\lVert A^{-1} \right\rVert \frac{\left\lVert r \right\rVert }{\left\lVert A \right\rVert } 
\]

Знаем что:
\[
  \left\lVert A \right\rVert \geq \frac{\left\lVert g \right\rVert }{\left\lVert v\right\rVert }
\]

Из этого получаем:
\[
  \left\lVert z \right\rVert \leq \left\lVert A \right\rVert \left\lVert A^{-1} \right\rVert
  \frac{\left\lVert r \right\rVert }{\left\lVert A \right\rVert } \left\lVert v \right\rVert 
\]
\[
  cond(A) = \left\lVert A \right\rVert \left\lVert A^{-1} \right\rVert
\]
\[
  \frac{\left\lVert r \right\rVert }{\left\lVert A \right\rVert } \sim \varepsilon_{M}
\]


\[
  \left\lVert z \right\rVert \leq cond(A) \varepsilon_{M} \left\lVert v \right\rVert 
\]

\subsubsection{Разложение невязки}

Разностная схема

\[
  -\left[ \frac{1}{r} \frac{d}{dr} \left(rk\frac{du}{dr} \right)\ -\ qu(r) \right] = f_i
\]
\[
  -\left[ \frac{d}{dr} \left(rk\frac{du}{dr} \right)\ -\ rqu(r) \right] = rf_i
\]

Введем новые обозначения:
\begin{center}
  $ \tilde{k} = rk \qquad \tilde{q} = rq \qquad \tilde{f} = rf $
\end{center}

Получим:
\[
  -\left[ \frac{d}{dr} \left(\tilde{k}\frac{du}{dr} \right)\ -\ \tilde{q}u(r) \right] = \tilde{f}_i
\]

Еще раз напишем разностную схему:
\[
  -\left[ \tilde{k}_{i+0.5}\frac{u_{i+1}-u_i}{h} + \nu_1 - \frac{h}{2} \tilde{q}_i v_i \right]\ =\ \frac{h}{2} \tilde{f}_i \quad i = 0
\]
\[
-\left[ \tilde{k}_{i+0.5}\frac{u_{i+1}-u_i}{h} - \tilde{k}_{i-0.5}\frac{u_{i} - u_{i-1}}{h} - h \tilde{q}_i v_i\right]\ =\ h \tilde{f}_i \quad i = 1, 2, ..., N -1
\]
\[
-\left[ \nu_2 - \tilde{k}_{i-0.5} \cdot \frac{u_i-u_{i-1}}{h}- \frac{h}{2} \tilde{q}_i v_i \right]\ =\ \frac{h}{2} \tilde{f}_i \quad i = N
\]

На левой границе интервала уравнение для невзяки выглядит следующим образом:
\[
  \xi_i = \frac{h}{2} \tilde{f}_i + \left [ \tilde{k}_{i+0.5}\frac{u_{i+1}-u_i}{h} + \nu_1 - \frac{h}{2} \tilde{q}_i u_i \right ]
\]

Внутри интервала:
\[
  \xi_i = h \tilde{f}_i + \left [ \tilde{k}_{i+0.5}\frac{u_{i+1}-u_i}{h} - \tilde{k}_{i-0.5}\frac{u_{i} - u_{i-1}}{h} - h \tilde{q}_i u_i \right ]
\]

На правой границе:
\[
  \xi_i = \frac{h}{2} \tilde{f}_i + \left [ \nu_2 - \tilde{k}_{i-0.5}\frac{u_{i} - u_{i-1}}{h} -  \frac{h}{2} \tilde{q}_i u_i \right ]
\]

Найдем разложение разностной схемы

\[
  u_i = u(x_i + h) = u_i + h \frac{d u_i}{d r} + \frac{h^2 }{2} \frac{d^2 u_i}{d r^2}
  + \frac{h^3 }{6} \frac{d^4 u_i}{d r^3} + \frac{h^4 }{24} \frac{d^4 u_i}{d r^4} + \mathcal{O}(h^5)
\]
\[
  \frac{u_{i + 1} - u_{i}}{h} = \frac{d u_i}{dr} + \frac{h}{2} \frac{d^2 u_i}{dr^2} 
  + \frac{h^2}{6}\frac{d^3 u_i}{dr^3} + \frac{h^3}{24}\frac{d^4 u_i}{dr^4} + \mathcal{O}(h^4)
\]
\[
  \tilde{k}_{i+\frac{1}{2}} = \tilde{k}(x_i + \frac{h}{2}) = \tilde{k_i} + \frac{h^2 }{2}\frac{d \tilde{k_i}}{d r} + \frac{h^2 }{8}\frac{d^2 \tilde{k_i}}{d r^2}
  + \frac{h^3 }{48}\frac{d^3 \tilde{k_i}}{d r^3} + \mathcal{O}(h^3)
\]
Получаем:
\begin{align*}
  \tilde{k}_{i+\frac{1}{2}} \frac{u_{i + 1} - u_{i}}{h} = &\tilde{k_i} \frac{du_i}{dr} \\
  &+ h \left [ \frac{1}{2}\tilde{k_i} \frac{d^2 u_i}{d r^2} + \frac{1}{2}\frac{d \tilde{k}_i}{dr} \frac{d u_i}{d r} \right ] \\
  &+ h^2 \left [ \frac{1}{6}\tilde{k_i} \frac{d^3 u_i}{d r^3} + \frac{1}{4}\frac{d \tilde{k}_i}{dr} \frac{d^2 u_i}{d r^2} + \frac{1}{8}\frac{d^2 \tilde{k_i}}{dr^2} \frac{d u_i}{d r} \right ] \\
  &+ h^3 \left [ \frac{1}{24}\tilde{k_i} \frac{d^4 u_i}{d r^4} + \frac{1}{12}\frac{d \tilde{k_i}}{dr} \frac{d^3 u_i}{d r^3} + \frac{1}{16}\frac{d^2 \tilde{k_i}}{dr^2} \frac{d^2 u_i}{d r^2} + \frac{1}{48}\frac{d^3 \tilde{k_i}}{dr^3} \frac{d u_i}{d r}\right ] \\
  &+ \mathcal{O}(h^4)
\end{align*}

\[
  u_{i-1} = u(x_i - h) = u_i - h \frac{d u_i}{d r} + \frac{h^2 }{2} \frac{d^2 u_i}{d r^2}
  - \frac{h^3 }{6} \frac{d^4 u_i}{d r^3} + \frac{h^4 }{24} \frac{d^4 u_i}{d r^4} + \mathcal{O}(h^5)
\]
\[
  \frac{u_{i} - u_{i-1}}{h} = \frac{d u_i}{dr} - \frac{h}{2} \frac{d^2 u_i}{dr^2} 
  + \frac{h^2}{6}\frac{d^3 u_i}{dr^3} - \frac{h^3}{24}\frac{d^4 u_i}{dr^4} + \mathcal{O}(h^4)
\]
\[
  \tilde{k}_{i-\frac{1}{2}} = \tilde{k}(x_i - \frac{h}{2}) = \tilde{k_i} - \frac{h^2 }{2}\frac{d \tilde{k_i}}{d r} + \frac{h^2 }{8}\frac{d^2 \tilde{k_i}}{d r^2}
  - \frac{h^3 }{48}\frac{d^3 \tilde{k_i}}{d r^3} + \mathcal{O}(h^3)
\]
Получаем:
\begin{align*}
  \tilde{k}_{i-\frac{1}{2}} \frac{u_{i} - u_{i-1}}{h} = &\tilde{k_i} \frac{du_i}{dr} \\
  &- h \left [ \frac{1}{2}\tilde{k_i} \frac{d^2 u_i}{d r^2} + \frac{1}{2}\frac{d \tilde{k}_i}{dr} \frac{d u_i}{d r} \right ] \\
  &+ h^2 \left [ \frac{1}{6}\tilde{k_i} \frac{d^3 u_i}{d r^3} + \frac{1}{4}\frac{d \tilde{k}_i}{dr} \frac{d^2 u_i}{d r^2} + \frac{1}{8}\frac{d^2 \tilde{k_i}}{dr^2} \frac{d u_i}{d r} \right ] \\
  &- h^3 \left [ \frac{1}{24}\tilde{k_i} \frac{d^4 u_i}{d r^4} + \frac{1}{12}\frac{d \tilde{k_i}}{dr} \frac{d^3 u_i}{d r^3} + \frac{1}{16}\frac{d^2 \tilde{k_i}}{dr^2} \frac{d^2 u_i}{d r^2} + \frac{1}{48}\frac{d^3 \tilde{k_i}}{dr^3} \frac{d u_i}{d r}\right ] \\
  &+ \mathcal{O}(h^4)
\end{align*}

Подставим это в уравнение невязки внутри интервала:
\begin{align*}
  \xi_i = h \tilde{f}_i +
  &\left [ \tilde{k_i} \frac{du_i}{dr} \right . + h \left [ \frac{1}{2}\tilde{k_i} \frac{d^2 u_i}{d r^2} + \frac{1}{2}\frac{dk_i}{dr} \frac{d u_i}{d r} \right ] 
    + h^2 \left [ \frac{1}{6}\tilde{k_i} \frac{d^3 u_i}{d r^3} + \frac{1}{4}\frac{dk_i}{dr} \frac{d^2 u_i}{d r^2} + \frac{1}{8}\frac{d^2 \tilde{k_i}}{dr^2} \frac{d u_i}{d r} \right ] \\
    &+ h^3 \left [ \frac{1}{24}\tilde{k_i} \frac{d^4 u_i}{d r^4} + \frac{1}{12}\frac{d \tilde{k_i}}{dr} \frac{d^3 u_i}{d r^3} + \frac{1}{16}\frac{d^2 \tilde{k_i}}{dr^2} \frac{d^2 u_i}{d r^2} + \frac{1}{48}\frac{d^3 \tilde{k_i}}{dr^3} \frac{d u_i}{d r} \right ] \\
    &- \tilde{k_i} \frac{du_i}{dr} + h \left [ \frac{1}{2}\tilde{k_i} \frac{d^2 u_i}{d r^2} + \frac{1}{2}\frac{dk_i}{dr} \frac{d u_i}{d r} \right ] \\
    &- h^2 \left [ \frac{1}{6}\tilde{k_i} \frac{d^3 u_i}{d r^3} + \frac{1}{4}\frac{dk_i}{dr} \frac{d^2 u_i}{d r^2} + \frac{1}{8}\frac{d^2 \tilde{k_i}}{dr^2} \frac{d u_i}{d r} \right ] \\
    &- h^3 \left [ \frac{1}{24}\tilde{k_i} \frac{d^4 u_i}{d r^4} + \frac{1}{12}\frac{d \tilde{k_i}}{dr} \frac{d^3 u_i}{d r^3} + \frac{1}{16}\frac{d^2 \tilde{k_i}}{dr^2} \frac{d^2 u_i}{d r^2} + \frac{1}{48}\frac{d^3 \tilde{k_i}}{dr^3} \left . \frac{d u_i}{d r}\right ] \right ]
\end{align*}

Приводим подобные слагаемые:
\begin{align*}
  \xi_i = &h \left [ \tilde{f} + \tilde{k} \frac{d^2u}{dr^2} \frac{d\tilde{k}}{dr}\frac{du}{dr} - \tilde{q}u \right ] \\
  &+ h^3 \left[ \frac{1}{12}\tilde{k} \frac{d^4 u_i}{d r^4}  + \frac{1}{6} \frac{d \tilde{k}}{dr} \frac{d^3 u_i}{d r^3} 
  + \frac{1}{8} \frac{d^2 \tilde{k}}{dr^2} \frac{d^3 u_i}{d r^3} + \frac{1}{24} \frac{d^3 \tilde{k}}{dr^3} \frac{d u_i}{d r} 
  \right] + \mathcal{O}(h^4)
\end{align*}

Можно заметить что:
\[
  \left [ \tilde{k} \frac{d^2u}{dr^2} + \frac{d\tilde{k}}{dr}\frac{du}{dr} \right ] =
  \frac{d}{dr}\left ( \tilde{k} \frac{du}{dr} \right )
\]

При этом:
\[
  \tilde{f} + \frac{d}{dr}\left ( \tilde{k} \frac{du}{dr} \right ) - \tilde{q}u = 0
\]

Тем самым получаем:
\begin{align*}
  \xi_i = h^3 \left[ \frac{1}{12}\tilde{k} \frac{d^4 u_i}{d r^4}  + \frac{1}{6} \frac{d \tilde{k}}{dr} \frac{d^3 u_i}{d r^3} 
  + \frac{1}{8} \frac{d^2 \tilde{k}}{dr^2} \frac{d^3 u_i}{d r^3} + \frac{1}{24} \frac{d^3 \tilde{k}}{dr^3} \frac{d u_i}{d r} 
  \right] + \mathcal{O}(h^4)
\end{align*}

Найдем разложение невязки для граничного условия слева:
\[
  \xi_i = \frac{h}{2} \tilde{f}_i + \left [ \tilde{k}_{i+0.5}\frac{u_{i+1}-u_i}{h} + \nu_1 - \frac{h}{2} \tilde{q}_i u_i \right ] \quad i = 0
\]

Получаем:
\begin{align*}
  \xi_i = &\frac{h}{2} \tilde{f}_i + \tilde{k_i} \frac{du_i}{dr} + h \left [ \frac{1}{2}\tilde{k_i} \frac{d^2 u_i}{d r^2} + \frac{1}{2}\frac{d \tilde{k}_i}{dr} \frac{d u_i}{d r} \right ] \\
  &+ h^2 \left [ \frac{1}{6}\tilde{k_i} \frac{d^3 u_i}{d r^3} + \frac{1}{4}\frac{d \tilde{k}_i}{dr} \frac{d^2 u_i}{d r^2} + \frac{1}{8}\frac{d^2 \tilde{k_i}}{dr^2} \frac{d u_i}{d r} \right ]
  + \mathcal{O}(h^3) + \nu_1 - \frac{h}{2} \tilde{q}_i u_i
\end{align*}

Сгруппируем слагаемые при одинаковых степенях h:
\begin{align*}
  \xi_i = & h^0 \left[ \nu_1 + \tilde{k_i} \frac{du_i}{dr}\right]
  + \frac{h}{2} \left[ \tilde{f}_i + \tilde{k_i} \frac{d^2 u_i}{d r^2} + \frac{d \tilde{k}_i}{dr} \frac{d u_i}{d r} - \tilde{q}_i u_i \right] \\
  &+ h^2 \left [ \frac{1}{6}\tilde{k_i} \frac{d^3 u_i}{d r^3} + \frac{1}{4}\frac{d \tilde{k}_i}{dr} \frac{d^2 u_i}{d r^2} + \frac{1}{8}\frac{d^2 \tilde{k_i}}{dr^2} \frac{d u_i}{d r} \right ] + \mathcal{O}(h^3)
\end{align*}

Из условия:
\[
  k \left. \frac{du}{dr}\right\vert_{r = R_L} = -\nu _1
\]

Также:
\[
  \left [ \tilde{k} \frac{d^2u}{dr^2} + \frac{d\tilde{k}}{dr}\frac{du}{dr} \right ] =
  \frac{d}{dr}\left ( \tilde{k} \frac{du}{dr} \right )
\]
\[
  \tilde{f} + \frac{d}{dr}\left ( \tilde{k} \frac{du}{dr} \right ) - \tilde{q}u = 0
\]

Тем самым получаем:
\begin{align*}
  \xi_i = h^2 \left [ \frac{1}{6}\tilde{k_i} \frac{d^3 u_i}{d r^3} + \frac{1}{4}\frac{d \tilde{k}_i}{dr} \frac{d^2 u_i}{d r^2} + \frac{1}{8}\frac{d^2 \tilde{k_i}}{dr^2} \frac{d u_i}{d r} \right ] + \mathcal{O}(h^3)
\end{align*}

Теперь найдем разложение невязки для правой границы
\[
  \xi_i = \frac{h}{2} \tilde{f}_i + \left [ \nu_2 - \tilde{k}_{i-0.5}\frac{u_{i} - u_{i-1}}{h} - \frac{h}{2} \tilde{q}_i u_i \right ], \quad i = N
\]

Получаем:
\begin{align*}
  \xi_i = &\frac{h}{2} \tilde{f}_i - \tilde{k_i} \frac{du_i}{dr} + h \left [ \frac{1}{2}\tilde{k_i} \frac{d^2 u_i}{d r^2} + \frac{1}{2}\frac{d \tilde{k}_i}{dr} \frac{d u_i}{d r} \right ] \\
  &- h^2 \left [ \frac{1}{6}\tilde{k_i} \frac{d^3 u_i}{d r^3} + \frac{1}{4}\frac{d \tilde{k}_i}{dr} \frac{d^2 u_i}{d r^2} + \frac{1}{8}\frac{d^2 \tilde{k_i}}{dr^2} \frac{d u_i}{d r} \right ]
  + \mathcal{O}(h^3) + \nu_2 - \frac{h}{2} \tilde{q}_i u_i
\end{align*}

Сгруппируем слагаемые при одинаковых степенях h:
\begin{align*}
  \xi_i = &h^0 \left[ \nu_2 - \tilde{k_i} \frac{du_i}{dr} \right] + \frac{h}{2} \left[ \tilde{f}_i + \tilde{k_i} \frac{d^2 u_i}{d r^2} + \frac{d \tilde{k}_i}{dr} \frac{d u_i}{d r} - \tilde{q}_i u_i\right] \\
  &- h^2 \left [ \frac{1}{6}\tilde{k_i} \frac{d^3 u_i}{d r^3} + \frac{1}{4}\frac{d \tilde{k}_i}{dr} \frac{d^2 u_i}{d r^2} + \frac{1}{8}\frac{d^2 \tilde{k_i}}{dr^2} \frac{d u_i}{d r} \right ] + \mathcal{O}(h^3)
\end{align*}

Из условия:
\[
  -k \left. \frac{du}{dr}\right\vert_{r = R_R}\ =\ -\nu_2
\]

Также:
\[
  \left [ \tilde{k} \frac{d^2u}{dr^2} + \frac{d\tilde{k}}{dr}\frac{du}{dr} \right ] =
  \frac{d}{dr}\left ( \tilde{k} \frac{du}{dr} \right )
\]
\[
  \tilde{f} + \frac{d}{dr}\left ( \tilde{k} \frac{du}{dr} \right ) - \tilde{q}u = 0
\]

Тем самым получаем:
\begin{align*}
  \xi_i = - h^2 \left [ \frac{1}{6}\tilde{k_i} \frac{d^3 u_i}{d r^3} + \frac{1}{4}\frac{d \tilde{k}_i}{dr} \frac{d^2 u_i}{d r^2} + \frac{1}{8}\frac{d^2 \tilde{k_i}}{dr^2} \frac{d u_i}{d r} \right ] + \mathcal{O}(h^3)
\end{align*}

\subsubsection{Зависимость погрешности от числа разбиений}

Как было показано выше, невязка $\left\lVert \xi \right\rVert \sim h^2$, 
когда погрешность решения системы алгебраических уравнений $ \left\lVert z \right\rVert \sim \frac{1}{h^2} $.
\begin{center}
  \includegraphics[width=0.7\textwidth]{img/err.pdf}
\end{center}

Значит существует такое число разбиений где погрешность минимальная.
	\newpage

	\subsection{Тестирование}

\subsubsection{Метод прогонки}

Нижняя и верхняя диагональ заполняются случайными числами. Главная диагональ заполняется следующим образом:
\[
  c_i = \left\lvert a_i \right\rvert + \left\lvert b_i \right\rvert + 1
\]
Главная диагональ заполняется таким образом для уменьшения возможности появления плохо обусловленной матрицы.
Также случайными числами заполняется вектор решения $ x $.

Чтобы получить вектор $ g $, мы умножаем матрицу на вектор решения. После получения вектора $ g $
используем наш метод прогонки, чтобы получить вектор решения $ \tilde{x} $. Для прохождения теста, полученные данные должны
соответствовать следующему неравенству:
\[
  \left\lVert \tilde{x} - x \right\rVert < cond(A) \varepsilon_M \left\lVert x \right\rVert 
\]
Результат выполнения тестов:
\begin{center}
  \includegraphics[width=\textwidth]{img/test.pdf}
\end{center}

\subsubsection{Интегро-интерполяционный метод}

Выполним нашу программу на ряде входных параметров

\begin{table}[H]
  \centering
  \begin{tabular}{ c | *{8}c }
    \toprule
    Номер теста & k & q & u & f & $\nu_1 $ & $\nu_2 $ & $R_L$ & $R_R$\\
    \midrule
    1 & r & 1 & 2r + 3 & 2r - 1 & -2 & 4 & 1 & 2 \\
    \midrule
    2 & $r^2$ & 1 & $2r^2 + 3$ & $-14r^2 + 3$ & -4 & 32 & 1 & 2 \\
    \midrule
    3 & $r^3$ & 1 & $2r^3 + 3$ & $-36r^4 + 2r^3 + 3$ & -6 & 192 & 1 & 2 \\
    \bottomrule
  \end{tabular}
\end{table}

Результаты тестов:
  \begin{table}[H]
    \centering
    \begin{tabular}{c | c | c}
      \toprule
      N & $ \left\lVert \varepsilon \right\rVert  $, одинарная точность & $ \left\lVert \varepsilon \right\rVert  $, двойная точность \\
      \midrule
      8 & 2.45905e-03 & 2.46952e-03\\
      16 & 6.10828e-04 & 6.19519e-04\\
      32 & 1.39713e-04 & 1.55015e-04\\
      64 & 6.91414e-05 & 3.87621e-05\\
      128 & 6.19888e-05 & 9.69106e-06\\
      256 & 2.66743e-03 & 2.42279e-06\\
      512 & 9.03511e-03 & 6.05706e-07\\
      1024 & 9.06944e-03 & 1.51435e-07\\
      2048 & 4.98871e-01 & 3.78822e-08\\
      \bottomrule
    \end{tabular}
    \caption{Погрешность теста №1}
  \end{table}

  \begin{table}[H]
    \centering
    \begin{tabular}{c | c | c}
      \toprule
      N & $ \left\lVert \varepsilon \right\rVert  $, одинарная точность & $ \left\lVert \varepsilon \right\rVert  $, двойная точность \\
      \midrule
      8 & 1.49273e-01 & 1.49265e-01\\
      16 & 3.73564e-02 & 3.73383e-02\\
      32 & 9.32503e-03 & 9.33596e-03\\
      64 & 1.68228e-03 & 2.33408e-03\\
      128 & 1.65176e-03 & 5.83525e-04\\
      256 & 1.83487e-03 & 1.45882e-04\\
      512 & 2.59638e-02 & 3.64704e-05\\
      1024 & 7.11317e-01 & 9.11763e-06\\
      2048 & 5.93484 & 2.27931e-06\\
      \bottomrule
    \end{tabular}
    \caption{Погрешность теста №2}
  \end{table}

  \begin{table}[H]
    \centering
    \begin{tabular}{c | c | c}
      \toprule
      N & $ \left\lVert \varepsilon \right\rVert  $, одинарная точность & $ \left\lVert \varepsilon \right\rVert  $, двойная точность \\
      \midrule
      8 & 2.52285 & 2.52289\\
      16 & 6.31502e-01 & 6.31225e-01\\
      32 & 1.5601e-01 & 1.57838e-01\\
      64 & 4.27036e-02 & 3.94614e-02\\
      128 & 1.83411e-02 & 9.86546e-03\\
      256 & 7.73621e-03 & 2.46637e-03\\
      512 & 2.19547e-01 & 6.16594e-04\\
      1024 & 1.32325 & 1.54148e-04\\
      2048 & 17.763 & 3.85369e-05\\
      \bottomrule
    \end{tabular}
    \caption{Погрешность теста №3}
  \end{table}

  Добавим еще один тест
  \begin{align*}
    k &= \sin(r) \quad q = \cos(r) \quad u = \tg(r) \\
    r &\in [\pi; \frac{3 \cdot \pi}{2} ]
  \end{align*}

  Находим значение $f$ как:
  \[
    -\left[ \frac{1}{r} \frac{d}{dr} \left(r \cdot \sin(r) \cdot \frac{d \tg(r)}{dr} \right)\ -\ \cos(r) \cdot \tg(r) \right] = f(r),
  \]

  Получаем:
  \[
    f(r) = \frac{\cos(r)^2}{r} + \sin(2 \cdot r) + \frac{\sin(r)^2}{\cos(r)}
  \]

  Находим значения $ \nu_1 $ и $ \nu_2 $:
  \begin{align*}
		&sin(r) \left. \frac{d\tg(r)}{dr}\right\vert_{r = \pi} = -\nu _1
		&-sin(r) \left. \frac{d\tg(r)}{dr}\right\vert_{r = \frac{3 \cdot \pi}{2}} = -\nu_2 \\
    & \nu_1 = 1 \\
    &\nu_2 = 0
  \end{align*}
  \begin{table}[H]
    \centering
    \begin{tabular}{c | c | c}
      \toprule
      N & $ \left\lVert \varepsilon \right\rVert  $, одинарная точность & $ \left\lVert \varepsilon \right\rVert  $, двойная точность \\
      \midrule
      8 & 1.1626e-02 & 1.16233e-02\\
      16 & 4.26462e-03 & 4.26086e-03\\
      32 & 9.20491e-02 & 9.30977e-02\\
      64 & 1.34945e-04 & 1.13523e-04\\
      128 & 5.553e-05 & 3.75747e-05\\
      256 & 3.72678e-05 & 1.3292e-05\\
      512 & 2.75075e-04 & 5.95552e-06\\
      1024 & 1.2811e-03 & 8.8951e-07\\
      2048 & 1.11288e-02 & 1.11339e-07\\
      \bottomrule
    \end{tabular}
    \caption{Погрешность теста №4}
  \end{table}
	\newpage

	\section{Заключение}
	\subsection{Вывод}
	Задание выполнено в полном объеме.
	Был написан написан метод приближенного вычисления краевой задачи и метод прогонки, а также написаны тесты к ним.
	Была оценена погрешность и выявлена зависимость погрешности от числа разбиения.
	\newpage
	\subsection{Код}
	\inputminted{c++}{../lab1/src/main.cpp}
	\inputminted{c++}{../lab1/src/data_table.hpp}
	\inputminted{c++}{../lab1/include/tma.hpp}
	\inputminted{c++}{../lab1/include/balance.hpp}
	\inputminted{c++}{../lab1/include/utils.hpp}
	\inputminted{c++}{../lab1/include/utils/balance_utils.hpp}
	\inputminted{c++}{../lab1/include/utils/data.hpp}
	\inputminted{c++}{../lab1/include/utils/grid.hpp}
	\inputminted{c++}{../lab1/test/test_tma.cpp}
	\inputminted{c++}{../lab1/test/test_utils.hpp}
\end{document}