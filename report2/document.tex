\documentclass[a4paper,12pt]{article}

\usepackage[T2A]{fontenc}
\usepackage[utf8]{inputenc}
\usepackage[english,russian]{babel}
\usepackage{amsmath, amsfonts, amsthm}
\usepackage{listings}
\usepackage{enumerate}
\usepackage{float}
\usepackage{graphicx}
\usepackage{nameref}
\usepackage{hyperref}
\usepackage{tabularx}
\usepackage{indentfirst}
\usepackage{booktabs}

\usepackage[newfloat]{minted}

\setminted{
  frame=lines,
  framesep=2mm,
  baselinestretch=1.2,
  fontsize=\footnotesize,
  linenos
}

\usepackage[top=3cm,bottom=3cm,left=2.5cm,right=2.5cm]{geometry}
\hypersetup{
	colorlinks=true,
	linkcolor=blue,
	filecolor=magenta,
	urlcolor=cyan
}
\urlstyle{same}

\begin{document}
	\begin{titlepage}	% начало титульной страницы

	\begin{center}		% выравнивание по центру

		\large Санкт-Петербургский политехнический университет Петра Великого\\
		\large Институт компьютерных наук и технологий\\
		\large Высшая школа программной инженерии \\[6cm]
		% название института, затем отступ 6см

    \huge Курсовая Работа\\[0.5cm] % название работы, затем отступ 0,5см
		\large по дисциплине\\[0.1cm]
		\large <<Математические модели>>

	\end{center}

		\noindent\large Выполнил: \hfill \large Ферапонтов М.В.\\
		\noindent\large Группа: \hfill \large гр. 3530904/00104\\

		\noindent\large Проверил: \hfill \large Воскобойников С. П.

	\vfill % заполнить всё доступное ниже пространство

	\begin{center}
	\large Санкт-Петербург\\
	\large \the\year % вывести дату
	\end{center} % закончить выравнивание по центру

\end{titlepage} % конец титульной страницы

\vfill % заполнить всё доступное ниже пространство

	\newpage
	\tableofcontents
	\newpage

	\section{Вступление}
	\subsection{Постановка задачи}

	Вариант CP. Используя интегро-интерполяционный метод (метод баланса), разработать программу для моделирования нестационарного распределения температуры в полом цилиндре, описываемого математической моделью вида:
	\begin{align*}
		&\frac{\partial u}{\partial t} = \frac{1}{r} \frac{\partial}{\partial r}
		\left ( rk(r, t)\frac{\partial u}{\partial r} \right ) - q(r, t)u + f(r,t),
		\ r \in \left[ R_L,\ R_R\right],\ t \in [0, T],
		\\
		&0 < c_1 \leq k(r, t) \leq c_2,\ 0 \leq q(r, t)
	\end{align*}

	Начальное условие:
	\[
		\left. u \right\vert_{t=0} = \varphi(r)
	\]

	Граничные условия: \newline
	\begin{align*}
		&k \left. \frac{\partial u}{\partial r}\right\vert_{r = R_L} = -\nu_1(t)
		&-k \left. \frac{\partial u}{\partial r}\right\vert_{r = R_R} = -\nu_2(t)
	\end{align*}
	\subsection{Используемое ПО}

	\begin{enumerate}
		\item \href{https://www.boost.org/}{Boost library} - библиотека для тестирования и других функций
		\item \href{https://www.gnu.org/software/gsl/}{\textbf{GSL}} - GNU Scientific Library. Математическая библиотека для C и C++.
	\end{enumerate}
	\newpage

	\section{Основная часть}
	\subsection{Интегро-интеполяционный метод (метод баланса)}

Введем основную сетку, где N - число разбиений.
\[r_0 < r_1 < \dots < r_N,\ r_i \in [R_L, R_R],\ r_0 = R_L,\ r_N = R_R\]
\[
  h_i =r_i - r_{i-1},\ i=1,2, \dots, N
\]
\[
  r_{r-0.5} = \frac{r_i - r_{i-1}}{2},\ i=1,2, \dots, N
\]
Введем дополнительную сетку:
\[
  \hbar_i = \begin{cases}
    \frac{h_i + 1}{2},\ i = 0 \\ \\
    \frac{h_i + h_{i+1}}{2},\ i = 1, 2, \dots, N-1 \\ \\
    \frac{h_i}{2},\ i = N
  \end{cases}
\]
Проведем аппроксимацию начального уравнения.\\
Для i\ =\ 0
\[ -\int_{r_i}^{r_{i+0.5}} \left[\frac{d}{dr}\left(r k(r) \frac{du(r)}{dr}\right) -r q(r)u(r)\right]  \,dr = \int_{r_i}^{r_{i+0.5}} r f(r) \,dr, \]
\[	-\left[r k(r)\left.\frac{du(r)}{dr} \right\vert_{r=r_{i+0.5}} - r k(r) \left. \frac{du(r)}{dr}\right\vert_{r_i} - \int_{r_i}^{r_{i+0.5}} r q(r) u(r)  \,dr \right]
  = \int_{r_i}^{r_{i+0.5}} r f(r) \,dr, \]
Формула центральных разностей:
\[	\frac{du(r)}{dr}\vert_{r = r_{i+0.5}}\ \approx \frac{v_{i+1}\ -\ v_i}{h_{i+1}}, \]
Граничное условие:
\[	k(r) \left. \frac{du(r)}{dr}\right\vert_{r=R_L}\ =\ -\nu_1, \]
Формула левых прямоугольников:
\[	\int_{r_i}^{r_{r+0.5}} r \varphi_i \,dr\ =\ \hbar_i r_i \varphi_i \]
Получаем разностную схему для i = 0:
\[
  -\left[ r_{i+0.5} \cdot k_{i+0.5}\frac{v_{i+1}-v_i}{h_{i+1}} - r_i \cdot (-\nu_1) - \hbar r_i q_i v_i \right]\ =\ \hbar_ir_if_i
\]
Для i\ =\ 1, 2, \dots, N-1
\[ -\int_{r_{i-0,5}}^{r_{i+0.5}} \left[\frac{d}{dr}\left(r k(r) \frac{du(r)}{dr}\right) -r q(r)u(r)\right]  \,dr = \int_{r_{i-0.5}}^{r_{i+0.5}} r f(r) \,dr, \]
\[	-\left[r k(r)\left.\frac{du(r)}{dr} \right\vert_{r=r_{i+0.5}} - r k(r) \left. \frac{du(r)}{dr}\right\vert_{r_{i-0.5}} - \int_{r_{i-0.5}}^{r_{i+0.5}} r q(r) u(r)  \,dr \right]
= \int_{r_{i-0.5}}^{r_{i+0.5}} r f(r) \,dr, \]
\[\frac{du(r)}{dr}\vert_{r = r_{i-0.5}}\ \approx \frac{v_{i}\ -\ v_{i-1}}{h_{i}}\]
\[ \int_{r_{i-0.5}}^{r_{r+0.5}} r \varphi_i \,dr\ =\ \hbar_i r_i \varphi_i \]
Получаем разностную схему для i\ =\ 1, 2, \dots, N-1:
\[
  -\left[ r_{i+0.5} \cdot k_{i+0.5}\frac{v_{i+1}-v_i}{h_{i+1}} - r_{i-0.5}k_{i-0.5}\frac{v_{i}\ -\ v_{i-1}}{h_{i}} - \hbar r_i q_i v_i\right]\ =\ \hbar_ir_if_i
\]
Для i\ =\ N:
  \[-\int_{r_{i-0.5} }^{r_{i}} \left[\frac{d}{dr}\left(r k(r) \frac{du(r)}{dr}\right) -r q(r)u(r)\right]  \,dr = \int_{r_{i-0.5}}^{r_{i}} r f(r) \,dr, \]
  \[-\left[r k(r)\left.\frac{du(r)}{dr} \right\vert_{r=r_{i}} - r k(r) \left. \frac{du(r)}{dr}\right\vert_{r_{i-0.5}} - \int_{r_{i-0.5}}^{r_i} r q(r) u(r)  \,dr \right]
  = \int_{r_{i-0.5}}^{r_i} r f(r) \,dr, \]
  \[\frac{du(r)}{dr}\vert_{r = r_{i-0.5}}\ \approx \frac{v_{i}\ -\ v_{i-1}}{h_{i}}, \]
  \[-k(r)\left. \frac{du(r)}{dr} \right\vert_{r=R_R}\ =\ -\nu_2 \]
  \[\int_{r_{i-0.5}}^{r_i} r \varphi(r) \,dr \approx \hbar_i r_i \varphi_i \]
Получаем разностную схему для i=N:
\[
  -\left[ -r_i \cdot (-\nu_2) - r_{i-0.5}k_{i-0.5} \cdot \frac{v_i-v_{i-1}}{h_i}- \hbar_ir_iq_iu_i \right]\ =\ \hbar_ir_if_i
\]

Сгрупиируем полученные уравнения:

\[
  -\left[ r_{i+0.5} \cdot k_{i+0.5}\frac{v_{i+1}-v_i}{h_{i+1}} - r_i \cdot (-\nu_1) - \hbar r_i q_i v_i \right]\ =\ \hbar_ir_if_i \quad i = 0
\]
\[
-\left[ r_{i+0.5} \cdot k_{i+0.5}\frac{v_{i+1}-v_i}{h_{i+1}} - r_{i-0.5}k_{i-0.5}\frac{v_{i}\ -\ v_{i-1}}{h_{i}} - \hbar r_i q_i v_i\right]\ =\ \hbar_ir_if_i \quad i = 1, 2, ..., N -1
\]
\[
-\left[ -r_i \cdot (-\nu_2) - r_{i-0.5}k_{i-0.5} \cdot \frac{v_i-v_{i-1}}{h_i}- \hbar_ir_iq_iu_i \right]\ =\ \hbar_ir_if_i \quad i = N
\]

После аппроксимации уравнения можно представить в виде системы из трёхдиагональной матрицы где a, c, b - диагонали матрица A и вектора g. Элементы матрицы:\newline
Для i\ =\ 0
\begin{align*}
  c_i = r_{i+0.5}\frac{k_{i+0.5}}{h_{i+1}} + \hbar_ir_iq_i \quad
  b_i = -r_{i+0.5} \cdot \frac{k_{i+0.5}}{h_{i+1}} \quad
  g_i = \hbar_ir_if_i + r(-\nu_1)
\end{align*}
Для i\ =\ 1, 2, \dots, N-1
\begin{align*}
  a_i &= -r_{i-0.5}\frac{k_{i-0.5}}{h_i} \quad
  c_i = r_{i-0.5}\frac{k_{i-0.5}}{h_i} + r_{i+0.5}\frac{k_{i+0.5}}{h_{i+1}} + \hbar_i r_iq_i \quad
  b_i = -r_{i+0.5}\frac{k_{i+0.5}}{h_{i+1}} \\
  g_i &= \hbar_i r_i f_i
\end{align*}
Для i\ =\ N:
\begin{align*}
  a_i = -r_{i-0.5}\frac{k_{i-0.5}}{h_i} \quad
  c_i = r_{i-0.5}\frac{k_{i-0.5}}{h_i} + \hbar_i r_iq_i \quad
  g_i = \hbar_i r_i f_i + r_i \cdot (-\nu_2)
\end{align*}

	\subsection{Явный метод Эйлера}
\subsection{Неявный метод Эйлера}

	\subsection{Тестирование}

Погрешность решения находится следующим образом:
\[
  \max_{t_0, \cdots, t_n} = \left\lVert \varepsilon(t) \right\rVert 
\]
\[
  \left\lVert \varepsilon(t) \right\rVert  = \left\lVert \tilde{v(t)} - v(t) \right\rVert 
\]

Для всех тестов возьмем одинаковое интервал для r и t.
\[
  r \in [\frac{\pi}{6}, \frac{\pi}{3}]
\]
\[
  T \in [0, 1]
\]

Для всех тестов возьмем одинаковые значения k и q:
\[
  k = \frac{\cos(r)}{2} + 3
\]
\[
  q = \frac{\sin(r)}{2} + 2
\]

Также для всех тестов возьмем одинаковое число разбиений:
\[
  N = 16
\]

\subsubsection{Стационарное решение}
\[
  u(r, t) = r
\]

Тогда получаем значение f, как:
\[
  f(r, t) = \frac{-\cos(r) + r \sin(r)-6 + r^2 \sin(r) + 4r^2}{2r}
\]

Значения $ \nu_1 $ и $ \nu_2 $ такие:
\[
  \nu_1 = \frac{\sqrt{3}}{4} + 3
\]
\[
  \nu_2 = 3.25
\]

После выполнения программы получаем следующие значения:
\begin{table}[H]
  \centering
  \begin{tabular}{c | c | c}
    \toprule
    H & $ \left\lVert \varepsilon \right\rVert  $, явный метод & $ \left\lVert \varepsilon \right\rVert  $, неявный метод \\
    \midrule
    1e-1 & 2.25e+22 & 5.05e-5\\
    1e-2 & 2.13e+140 & 5.05e-5\\
    1e-3 & 1.09e+330 & 5.05e-5\\
    1e-4 & 3.57e-10 & 5.05e-5\\
    1e-5 & 3.88e-9 & 5.05e-5\\
    1e-6 & 3.84e-8 & 5.05e-5\\
    \bottomrule
  \end{tabular}
  \caption{Погрешность теста ДУ со стационарномым решением}
\end{table}

\subsubsection{Решение стремящееся к стационарному}
\[
  u(r, t) = 4 + e^{-10t}
\]

Тогда получаем значение f, как:
\[
  f(r, t) = \frac{-16 + 4e^{10t}\sin(r) + \sin(r) + 16e^{10t}}{2e^{10t}}
\]

Значения $ \nu_1 $ и $ \nu_2 $ такие:
\[
  \nu_1 = 0
\]
\[
  \nu_2 = 0
\]

После выполнения программы получаем следующие значения:
\begin{table}[H]
  \centering
  \begin{tabular}{c | c | c}
    \toprule
    H & $ \left\lVert \varepsilon \right\rVert  $, явный метод & $ \left\lVert \varepsilon \right\rVert  $, неявный метод \\
    \midrule
    1e-1 & 2.96e+14 & 1.65e-01\\
    1e-2 & 4.55e+140 & 1.36e-02\\
    1e-3 & 4.52e+292 & 2.94e-03\\
    1e-4 & 2.76e-05 & 4.54e-04\\
    1e-5 & 2.76e-05 & 2.98e-04\\
    1e-6 & 2.76e-03 & 2.44e-04\\
    \bottomrule
  \end{tabular}
  \caption{Погрешность теста ДУ с решением стремящимся к стационарному}
\end{table}

\subsubsection{Нестационарное решение}
\[
  u(r, t) = r + t
\]

Тогда получаем значение f, как:
\[
  f(r, t) = 1 - \frac{\cos(r) - r\sin(r) + 6}{2r} + \frac{t\sin(r)+\sin(r)}{2} + 2t + 2r
\]

Значения $ \nu_1 $ и $ \nu_2 $ такие:
\[
  \nu_1 = \frac{\sqrt{3}}{4} + 3
\]
\[
  \nu_2 = 3.25
\]

После выполнения программы получаем следующие значения:
\begin{table}[H]
  \centering
  \begin{tabular}{c | c | c}
    \toprule
    H & $ \left\lVert \varepsilon \right\rVert  $, явный метод & $ \left\lVert \varepsilon \right\rVert  $, неявный метод \\
    \midrule
    1e-1 & 5.57e+20 & 3.18e-01\\
    1e-2 & 1.18e+139 & 3.3e-02\\
    1e-3 & 6.48e+300 & 3.31e-03\\
    1e-4 & 1.69e-05 & 3.57e-04\\
    1e-5 & 1.69e-05 & 1.29e-05\\
    1e-6 & 1.69e-05 & 8.48e-05\\
    \bottomrule
  \end{tabular}
  \caption{Погрешность теста ДУ с нестационарным решением}
\end{table}

Полученные данные дают нам повод убедиться, что явный метод Эйлера непременим для решения жестких систем.
При небольшом шаге можно заметить "взрыв погрешности". 

В свою же очередь неявный метод Эйлера требует решения системы линейных алгебраических уравнений, что требует
значительного увеличения числа вычислений.
	\newpage

	\section{Заключение}
	\subsection{Вывод}
	Задание выполнено полностью. Были написаны: метод для приближенного решения дифференциального уравнения, явный и неявный метод Эйлера.
	Все методы были протестированы. Были выявлены недостатки и преимущества явного и неявного метода Эйлера.
	\newpage
	\subsection{Код}
	\inputminted{c++}{../lab2/src/main.cpp}
	\inputminted{c++}{../lab2/src/data_table.hpp}
	\inputminted{c++}{../lab2/include/tma.hpp}
	\inputminted{c++}{../lab2/include/euler.hpp}
	\inputminted{c++}{../lab2/include/utils.hpp}
	\inputminted{c++}{../lab2/include/utils/balance_utils.hpp}
	\inputminted{c++}{../lab2/include/utils/data.hpp}
	\inputminted{c++}{../lab2/include/utils/grid.hpp}
\end{document}