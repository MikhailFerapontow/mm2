\documentclass[a4paper,12pt]{article}

\usepackage[T2A]{fontenc}
\usepackage[utf8]{inputenc}
\usepackage[english,russian]{babel}
\usepackage{fontspec}
\setmainfont[Scale=1.167]{Times New Roman}

\usepackage{unicode-math}

\usepackage{microtype}
\usepackage[linesnumbered,ruled,vlined]{algorithm2e}

\usepackage{amsmath, amsfonts, amsthm}
\usepackage{listings}
\usepackage{enumerate}
\usepackage{float}
\usepackage{graphicx}
\usepackage{nameref}
\usepackage{hyperref}
\usepackage{tabularx}
\usepackage{makecell}
\usepackage{indentfirst}
\usepackage{booktabs}
\usepackage{derivative}
\usepackage{physics}

\usepackage{tikz}
\usepackage{tikzscale}
\usetikzlibrary{decorations.pathreplacing,calligraphy}
\usetikzlibrary{arrows.meta}
\usetikzlibrary{backgrounds,fit,positioning}
\usetikzlibrary{external}

\usepackage[newfloat]{minted}

\setminted{
  frame=lines,
  framesep=2mm,
  baselinestretch=1.2,
  fontsize=\footnotesize,
  linenos,
  breaklines
}

\usepackage[top=3cm,bottom=3cm,left=2.5cm,right=2.5cm]{geometry}
\hypersetup{
	colorlinks=true,
	linkcolor=blue,
	filecolor=magenta,
	urlcolor=cyan
}
\urlstyle{same}

\begin{document}
	\begin{titlepage}	% начало титульной страницы

	\begin{center}		% выравнивание по центру

		\large Санкт-Петербургский политехнический университет Петра Великого\\
		\large Институт компьютерных наук и технологий\\
		\large Высшая школа программной инженерии \\[6cm]
		% название института, затем отступ 6см

    \huge Лабораторная работа № 1\\[0.5cm] % название работы, затем отступ 0,5см
		\large по дисциплине\\[0.1cm]
		\large <<Математические модели>>

	\end{center}

		\noindent\large Выполнил: \hfill \large Ферапонтов М.В.\\
		\noindent\large Группа: \hfill \large гр. 3530904/00104\\

		\noindent\large Проверил: \hfill \large Воскобойников С. П.

	\vfill % заполнить всё доступное ниже пространство

	\begin{center}
	\large Санкт-Петербург\\
	\large \the\year % вывести дату
	\end{center} % закончить выравнивание по центру

\end{titlepage} % конец титульной страницы

\vfill % заполнить всё доступное ниже пространство

	\newpage
	\tableofcontents
	\newpage

	\section{Вступление}
	\subsection{Постановка задачи}

	Вариант CP. Используя интегро-интерполяционный метод (метод баланса), разработать программу для моделирования нестационарного распределения температуры в полом цилиндре, описываемого математической моделью вида:
	\begin{align*}
		&\frac{\partial u}{\partial t} = \frac{1}{r} \frac{\partial}{\partial r}
		\left ( rk(r, t)\frac{\partial u}{\partial r} \right ) - q(r, t)u + f(r,t),
		\ r \in \left[ R_L,\ R_R\right],\ t \in [0, T],
		\\
		&0 < c_1 \leq k(r, t) \leq c_2,\ 0 \leq q(r, t)
	\end{align*}

	Начальное условие:
	\[
		\left. u \right\vert_{t=0} = \varphi(r)
	\]

	Граничные условия: \newline
	\begin{align*}
		&k \left. \frac{\partial u}{\partial r}\right\vert_{r = R_L} = -\nu_1(t)
		&-k \left. \frac{\partial u}{\partial r}\right\vert_{r = R_R} = -\nu_2(t)
	\end{align*}
	\subsection{Используемое ПО}

	\begin{enumerate}
		\item \href{https://www.boost.org/}{Boost library} - библиотека для тестирования и других функций
		\item \href{https://www.gnu.org/software/gsl/}{\textbf{GSL}} - GNU Scientific Library. Математическая библиотека для C и C++.
	\end{enumerate}
	\newpage

	\section{Основная часть}
	\subsection{Разностная схема}
Введем основную сетку:
\begin{align*}
  &N_r - \text{число разбиений на} [0, R] & &N_z - \text{число разбиений на} [0, L] \\
  &r_0 < r_1 < \cdots < r_N & &z_0 < z_1 < \cdots < z_N \\
  &r_0 = 0,\quad r_N = R & &z_0 = 0,\quad z_N = L \\
  &h_i = \frac{R - 0}{N_r},\quad i=0,\cdots, N_r & &h_j = \frac{L - 0}{N_z},\quad j=0,\cdots, N_z 
\end{align*}

Введем дополнительную сетку:
\begin{align*}
  &r_{i-\frac{1}{2}} = \frac{r_i + r_{i - 1}}{2}\quad i=1,\cdots, N_r & &z_{j-\frac{1}{2}} = \frac{z_j + z_{j - 1}}{2}\quad j=1,\cdots, N_z \\
  &  \hbar_i = \begin{cases}
    \frac{h_i + 1}{2},\ i = 0 \\ \\
    \frac{h_i + h_{i+1}}{2},\ i = 1, 2, \dots, N_r-1 \\ \\
    \frac{h_i}{2},\ i = N_r
  \end{cases} &
  &   \hbar_j = \begin{cases}
    \frac{h_j + 1}{2},\ j = 0 \\ \\
    \frac{h_j + h_{j+1}}{2},\ j = 1, 2, \dots, N_z-1 \\ \\
    \frac{h_j}{2},\ j = N_z
  \end{cases}
\end{align*}

Преобразуем наше начальное уравнение

\[
  - \left [ \pdv{}{r} \left ( r k_1(r, z) \pdv{u}{r} \right ) 
  + \pdv{}{z} \left ( rk_2(r, z)\pdv{u}{v} \right ) \right ] = rf(r, z)
\]

Проинтегрируем уравнение внутри интервала:

\[
  - \mLim{+\frac{1}{2}}{-\frac{1}{2}}{+\frac{1}{2}}{-\frac{1}{2}} \left [ \pdv{}{r} \left ( r k_1(r, z) \pdv{u}{r} \right ) 
  + \pdv{}{z} \left ( rk_2(r, z)\pdv{u}{v} \right ) \right ] dr dz = \mLim{+\frac{1}{2}}{-\frac{1}{2}}{+\frac{1}{2}}{-\frac{1}{2}} rf(r, z) dr dz
\]

Получим:

\begin{align*}
  &- \left [
   \mLimS{z}{+\frac{1}{2}}{-\frac{1}{2}}  \left . r k_1(r, z) \pdv{u}{r} \right \vert_{r = r_{i + \frac{1}{2}}} dz
  - \mLimS{z}{+\frac{1}{2}}{-\frac{1}{2}} \left . r k_1(r, z) \pdv{u}{r} \right \vert_{r = r_{i - \frac{1}{2}}} dz
  \right . \\
  &\left . + \mLimS{r}{+\frac{1}{2}}{-\frac{1}{2}} \left . rk_2(r, z)\pdv{u}{v} \right \vert_{z = z_{j + \frac{1}{2}}} dr
  - \mLimS{r}{+\frac{1}{2}}{-\frac{1}{2}} \left . rk_2(r, z)\pdv{u}{v} \right \vert_{z = z_{j - \frac{1}{2}}} dr
  \right ] = \mLim{+\frac{1}{2}}{-\frac{1}{2}}{+\frac{1}{2}}{-\frac{1}{2}} rf(r, z) dr dz
\end{align*}

Воспользуемся формулами численного дифференцирования:
\[
  \left . k_1(r, z) \pdv{u}{r} \right \vert_{r = r_{i - \frac{1}{2}}}
  \approx k_1(r_{i - \frac{1}{2}}, z) 
  \frac{v_{i, j} - v_{i - 1, j}}{h_i}
\]

\[
  \left . k_2(r, z) \pdv{u}{r} \right \vert_{z = z_{j - \frac{1}{2}}}
  \approx k_2(r, z_{j - \frac{1}{2}}) 
  \frac{v_{i, j} - v_{i, j - 1}}{h_j}
\]

Также воспользуемся формулой средних прямоугольников:
\[
  \mLimS{r}{+\frac{1}{2}}{-\frac{1}{2}} r \varphi(r, z) dr
  = \hbar_i r_i \varphi_i
\]
\[
  \mLim{+\frac{1}{2}}{-\frac{1}{2}}{+\frac{1}{2}}{-\frac{1}{2}} r \varphi(r, z) drdz
  = \hbar_i\hbar_j r_i \varphi_{i, j}
\]

В итоге получаем разностную схему внутри интервала:
\begin{align*}
  - \left [ 
  \hbar_j r_{i+\frac{1}{2}} k_1(r_{i+\frac{1}{2}}, z_j) \frac{v_{i+1, j} - v_{i, j}}{h_{i + 1}}
  - \hbar_j r_{i-\frac{1}{2}} k_1(r_{i-\frac{1}{2}}, z_j) \frac{v_{i, j} - v_{i - 1, j}}{h_{i}}
  \right . \\
  \left .
  + \hbar_i r_{i+\frac{1}{2}} k_2(r_i, z_{j+\frac{1}{2}}) \frac{v_{i, j + 1} - v_{i, j}}{h_{j + 1}}
  - \hbar_i r_{i-\frac{1}{2}} k_2(r_i, z_{j-\frac{1}{2}}) \frac{v_{i, j} - v_{i, j - 1}}{h_j}
  \right ]  = \hbar_i \hbar_j r_i f_{i, j}
\end{align*}

Теперь найдем значение разностной схемы на углах и границах интервалов

\subsubsection{На левой границе}

Проинтегрируем наше уравнение в $ i = 0 $ и $ z $ внутри промежутка
\[
  - \mLim{+\frac{1}{2}}{}{+\frac{1}{2}}{-\frac{1}{2}} \left [ \pdv{}{r} \left ( r k_1(r, z) \pdv{u}{r} \right ) 
  + \pdv{}{z} \left ( rk_2(r, z)\pdv{u}{v} \right ) \right ] dr dz = \mLim{+\frac{1}{2}}{}{+\frac{1}{2}}{-\frac{1}{2}} rf(r, z) dr dz
\]

Получаем:
\begin{align*}
  - \left [
   \mLimS{z}{+\frac{1}{2}}{-\frac{1}{2}}  \left . r k_1(r, z) \pdv{u}{r} \right \vert_{r = r_{i + \frac{1}{2}}} dz
  - \mLimS{z}{+\frac{1}{2}}{-\frac{1}{2}} \left . r k_1(r, z) \pdv{u}{r} \right \vert_{r = r_{i}} dz
  \right . \\
  \left . + \mLimS{r}{+\frac{1}{2}}{} \left . rk_2(r, z)\pdv{u}{v} \right \vert_{z = z_{j + \frac{1}{2}}} dr
  - \mLimS{r}{+\frac{1}{2}}{} \left . rk_2(r, z)\pdv{u}{v} \right \vert_{z = z_{j - \frac{1}{2}}} dr
  \right ] = \mLim{+\frac{1}{2}}{}{+\frac{1}{2}}{-\frac{1}{2}} rf(r, z) dr dz
\end{align*}

Имеем граничное условие:
\[
  \left . u \right \vert_{r=0} - \text{ограничено, т. е } \left . \frac{\partial u}{ \partial r} \right |_{r = 0} = 0
\]

\[
  \mLimS{r}{+ \frac{1}{2}}{} rf dr \approx f_i \mLimS{r}{ + \frac{1}{2}}{} r dr = 
  f_i \frac{r^2_{i + \frac{1}{2}}}{2} = h_r f_i \frac{r_{i + \frac{1}{2}}}{2},
  \quad i = 0, \quad r_i = 0, r_{i + \frac{1}{2}} = \frac{h_r}{2}
\]


Получаем разностную схему:
\begin{align*}
  &- \left [ 
  \hbar_j r_{i+\frac{1}{2}} k_1(r_{i+\frac{1}{2}}, z_j) \frac{v_{i+1, j} - v_{i, j}}{h_{i + 1}}
  - 0
  \right . \\
  &\left .
  + \hbar_i r_{i+\frac{1}{2}} k_2(r_i, z_{j+\frac{1}{2}}) \frac{v_{i, j + 1} - v_{i, j}}{h_{j + 1}}
  - \hbar_i r_{i-\frac{1}{2}} k_2(r_i, z_{j-\frac{1}{2}}) \frac{v_{i, j} - v_{i, j - 1}}{h_j}
  \right ]  = \hbar_i \hbar_j r_i f_{i, j}
\end{align*}

\subsubsection{На правой границе}
Проинтегрируем наше уравнение в $ i = N_x $ и $ z $ внутри промежутка
\[
  - \mLim{}{-\frac{1}{2}}{+\frac{1}{2}}{-\frac{1}{2}} \left [ \pdv{}{r} \left ( r k_1(r, z) \pdv{u}{r} \right ) 
  + \pdv{}{z} \left ( rk_2(r, z)\pdv{u}{v} \right ) \right ]dr dz = \mLim{}{-\frac{1}{2}}{+\frac{1}{2}}{-\frac{1}{2}} rf(r, z)dr dz
\]

Получаем:
\begin{align*}
  &- \left [
   \mLimS{z}{+\frac{1}{2}}{-\frac{1}{2}}  \left . r k_1(r, z) \pdv{u}{r} \right \vert_{r = r_{i}} dz
  - \mLimS{z}{+\frac{1}{2}}{-\frac{1}{2}} \left . r k_1(r, z) \pdv{u}{r} \right \vert_{r = r_{i - \frac{1}{2}}} dz
  \right . \\
  &\left . + \mLimS{r}{}{-\frac{1}{2}} \left . rk_2(r, z)\pdv{u}{v} \right \vert_{z = z_{j + \frac{1}{2}}} dr
  - \mLimS{r}{}{-\frac{1}{2}} \left . rk_2(r, z)\pdv{u}{v} \right \vert_{z = z_{j - \frac{1}{2}}} dr
  \right ] = \mLim{}{-\frac{1}{2}}{+\frac{1}{2}}{-\frac{1}{2}} rf(r, z) dr dz
\end{align*}

Имеем граничное условие:
\[
  \left . -k_1 \pdv{u}{r} \right \vert_{r=R} = \chi_2 \left . u \right \vert_{r=R} - \varphi_2(z)
\]

Получаем разностную схему:
\begin{align*}
  &- \left [ 
  -\hbar_j ( \chi_2 v_i - \varphi_2(z))
  - \hbar_j r_{i-\frac{1}{2}} k_1(r_{i-\frac{1}{2}}, z_j) \frac{v_{i, j} - v_{i - 1, j}}{h_{i}}
  \right . \\
  &\left .
  + \hbar_i r_{i+\frac{1}{2}} k_2(r_i, z_{j+\frac{1}{2}}) \frac{v_{i, j + 1} - v_{i, j}}{h_{j + 1}}
  - \hbar_i r_{i-\frac{1}{2}} k_2(r_i, z_{j-\frac{1}{2}}) \frac{v_{i, j} - v_{i, j - 1}}{h_j}
  \right ]  = \hbar_i \hbar_j r_i f_{i, j}
\end{align*}

\subsubsection{На нижней границе}
Проинтегрируем наше уравнение $ j = 0 $ и $ i $ внутри промежутка
\[
  - \mLim{+\frac{1}{2}}{-\frac{1}{2}}{+\frac{1}{2}}{} \left [ \pdv{}{r} \left ( r k_1(r, z) \pdv{u}{r} \right ) 
  + \pdv{}{z} \left ( rk_2(r, z)\pdv{u}{v} \right ) \right ] drdz = \mLim{+\frac{1}{2}}{-\frac{1}{2}}{+\frac{1}{2}}{} rf(r, z) dr dz
\]

Получаем:
\begin{align*}
  &- \left [
   \mLimS{z}{+\frac{1}{2}}{}  \left . r k_1(r, z) \pdv{u}{r} \right \vert_{r = r_{i + \frac{1}{2}}} dz
  - \mLimS{z}{+\frac{1}{2}}{} \left . r k_1(r, z) \pdv{u}{r} \right \vert_{r = r_{i - \frac{1}{2}}} dz
  \right . \\
  &\left . + \mLimS{r}{+\frac{1}{2}}{-\frac{1}{2}} \left . rk_2(r, z)\pdv{u}{v} \right \vert_{z = z_{j + \frac{1}{2}}} dr
  - \mLimS{r}{+\frac{1}{2}}{-\frac{1}{2}} \left . rk_2(r, z)\pdv{u}{v} \right \vert_{z = z_{j}} dr
  \right ] = \mLim{+\frac{1}{2}}{-\frac{1}{2}}{+\frac{1}{2}}{} rf(r, z) dr dz
\end{align*}

Имеем граничное условие:
\[
  \left . k_2 \pdv{u}{z} \right \vert_{z=0} = \chi_3 \left . u \right \vert_{z=0} - \varphi_3(r)
\]

Получаем разностную схему:
\begin{align*}
  - \left [ 
  \hbar_j r_{i+\frac{1}{2}} k_1(r_{i+\frac{1}{2}}, z_j) \frac{v_{i+1, j} - v_{i, j}}{h_{i + 1}}
  - \hbar_j r_{i-\frac{1}{2}} k_1(r_{i-\frac{1}{2}}, z_j) \frac{v_{i, j} - v_{i - 1, j}}{h_{i}}
  \right . \\
  \left .
  + \hbar_i r_{i+\frac{1}{2}} k_2(r_i, z_{j+\frac{1}{2}}) \frac{v_{i, j + 1} - v_{i, j}}{h_{j + 1}}
  - \hbar_i(\chi_3 v_i - \varphi_3(r))
  \right ]  = \hbar_i \hbar_j r_i f_{i, j}
\end{align*}

\subsubsection{На верхней границе}
Имеем граничное условие:
\[
  \left . u \right \vert_{z=L} = \varphi_r(r) 
\]

\subsubsection{Правый-нижний угол}

Проинегрируем наше уравнение
\[
  - \mLim{}{-\frac{1}{2}}{+\frac{1}{2}}{} \left [ \pdv{}{r} \left ( r k_1(r, z) \pdv{u}{r} \right ) 
  + \pdv{}{z} \left ( rk_2(r, z)\pdv{u}{v} \right ) \right ] = \mLim{}{-\frac{1}{2}}{+\frac{1}{2}}{} rf(r, z)
\]

Получаем:

\begin{align*}
  &- \left [
   \mLimS{z}{+\frac{1}{2}}{}  \left . r k_1(r, z) \pdv{u}{r} \right \vert_{r = r_i} dz
  - \mLimS{z}{+\frac{1}{2}}{} \left . r k_1(r, z) \pdv{u}{r} \right \vert_{r = r_{i - \frac{1}{2}}} dz
  \right . \\
  &\left . + \mLimS{r}{}{-\frac{1}{2}} \left . rk_2(r, z)\pdv{u}{v} \right \vert_{z = z_{j + \frac{1}{2}}} dr
  - \mLimS{r}{}{-\frac{1}{2}} \left . rk_2(r, z)\pdv{u}{v} \right \vert_{z = z_{j}} dr
  \right ] = \mLim{}{-\frac{1}{2}}{+\frac{1}{2}}{} rf(r, z) dr dz
\end{align*}

Имеем граничные условия:

\begin{align*}
  &\left . -k_1 \pdv{u}{r} \right \vert_{r=R} = \chi_2 \left . u \right \vert_{r=R} - \varphi_2(z) \\
  &\left . k_2 \pdv{u}{z} \right \vert_{z=0} = \chi_3 \left . u \right \vert_{z=0} - \varphi_3(r) 
\end{align*}

Получаем разностную схему:

\begin{align*}
  - \left [ 
  -\hbar_j (\chi_2 \left . u \right \vert_{r=R} - \varphi_2(z) )
  - \hbar_j r_{i-\frac{1}{2}} k_1(r_{i-\frac{1}{2}}, z_j) \frac{v_{i, j} - v_{i - 1, j}}{h_{i}}
  \right . \\
  \left .
  + \hbar_i r_{i+\frac{1}{2}} k_2(r_i, z_{j+\frac{1}{2}}) \frac{v_{i, j + 1} - v_{i, j}}{h_{j + 1}}
  - \hbar_i(\chi_3 v_i - \varphi_3(r))
  \right ]  = \hbar_i \hbar_j r_i f_{i, j}
\end{align*}

Другие углы нам известны.

% \subsubsection{Правый-верхний угол}
% \[
%   - \mLim{}{-\frac{1}{2}}{}{-\frac{1}{2}} \left [ \pdv{}{r} \left ( r k_1(r, z) \pdv{u}{r} \right ) 
%   + \pdv{}{z} \left ( rk_2(r, z)\pdv{u}{v} \right ) \right ] = \mLim{}{-\frac{1}{2}}{}{-\frac{1}{2}} rf(r, z)
% \]

	\subsubsection{Явный метод Эйлера}

Мы имеем уравнение:
\[
  v(t_{n+1}) = v(t_n) + \int\limits_{t_{n}}^{t_n+1} (Av + g) dt
\]

Интерполлируем интеграл по формуле левых треугольников:
\[
  \int\limits_{a}^{b} f(x) \mathrm{d} x \approx f(a) (b - a)
\]

Тем самым получаем:
\[
  v(t_{n+1}) = v(t_n) + (t_{n+1} - t_n) Av(t_n) + (t_{n+1} - t_n) g
\]

Введем обозначение:
\[
  H = (t_{n+1} - t_n)
\]
\[
  v(t_{n+1}) = v(t_n) + H Av(t_n) + H g
\]
\[
  v(t_{n+1}) = (E + HA)v(t_n) + Hg
\]

Явный метод ломанных Эйлера:
\[
  \begin{cases}
    v(t_{n+1}) = (E + HA)v(t_n) + Hg \\
    v(t_0) = \varphi(r)
  \end{cases}
\]

\subsubsection{Неявный метод Эйлера}

Теперь проинтерполируем интеграл формулой правых треугольников:
\[
  \int\limits_{a}^{b} f(x) \mathrm{d} x \approx f(b) (b - a)
\]

Получаем:
\[
  v(t_{n+1}) = v(t_n) + HAv(t_{n+1}) + Hg
\]
\[
  (E - HA) v(t_{t_{n+1}}) = v(t_n) + Hg
\]

Неявный метод ломанных Эйлера:
\[
  \begin{cases}
    (E - HA) v(t_{t_{n+1}}) = v(t_n) + Hg \\
    v(t_0) = \varphi(r)
  \end{cases}
\]
	\subsection{Тестирование}

Погрешность решения находится следующим образом:
\[
  \max_{t_0, \cdots, t_n} = \left\lVert \varepsilon(t) \right\rVert 
\]
\[
  \left\lVert \varepsilon(t) \right\rVert  = \left\lVert \tilde{v(t)} - v(t) \right\rVert 
\]

Для всех тестов возьмем одинаковое интервал для r и t.
\[
  r \in [\frac{\pi}{6}, \frac{\pi}{3}]
\]
\[
  T \in [0, 1]
\]

Для всех тестов возьмем одинаковые значения k и q:
\[
  k = \frac{\cos(r)}{2} + 3
\]
\[
  q = \frac{\sin(r)}{2} + 2
\]

Также для всех тестов возьмем одинаковое число разбиений:
\[
  N = 16
\]

\subsubsection{Стационарное решение}
\[
  u(r, t) = r
\]

Тогда получаем значение f, как:
\[
  f(r, t) = \frac{-\cos(r) + r \sin(r)-6 + r^2 \sin(r) + 4r^2}{2r}
\]

Значения $ \nu_1 $ и $ \nu_2 $ такие:
\[
  \nu_1 = \frac{\sqrt{3}}{4} + 3
\]
\[
  \nu_2 = 3.25
\]

После выполнения программы получаем следующие значения:
\begin{table}[H]
  \centering
  \begin{tabular}{c | c | c}
    \toprule
    H & $ \left\lVert \varepsilon \right\rVert  $, явный метод & $ \left\lVert \varepsilon \right\rVert  $, неявный метод \\
    \midrule
    1e-1 & 2.25e+22 & 5.05e-5\\
    1e-2 & 2.13e+140 & 5.05e-5\\
    1e-3 & 1.09e+330 & 5.05e-5\\
    1e-4 & 3.57e-10 & 5.05e-5\\
    1e-5 & 3.88e-9 & 5.05e-5\\
    1e-6 & 3.84e-8 & 5.05e-5\\
    \bottomrule
  \end{tabular}
  \caption{Погрешность теста ДУ со стационарномым решением}
\end{table}

\subsubsection{Решение стремящееся к стационарному}
\[
  u(r, t) = 4 + e^{-10t}
\]

Тогда получаем значение f, как:
\[
  f(r, t) = \frac{-16 + 4e^{10t}\sin(r) + \sin(r) + 16e^{10t}}{2e^{10t}}
\]

Значения $ \nu_1 $ и $ \nu_2 $ такие:
\[
  \nu_1 = 0
\]
\[
  \nu_2 = 0
\]

После выполнения программы получаем следующие значения:
\begin{table}[H]
  \centering
  \begin{tabular}{c | c | c}
    \toprule
    H & $ \left\lVert \varepsilon \right\rVert  $, явный метод & $ \left\lVert \varepsilon \right\rVert  $, неявный метод \\
    \midrule
    1e-1 & 2.96e+14 & 1.65e-01\\
    1e-2 & 4.55e+140 & 1.36e-02\\
    1e-3 & 4.52e+292 & 2.94e-03\\
    1e-4 & 2.76e-05 & 4.54e-04\\
    1e-5 & 2.76e-05 & 2.98e-04\\
    1e-6 & 2.76e-03 & 2.44e-04\\
    \bottomrule
  \end{tabular}
  \caption{Погрешность теста ДУ с решением стремящимся к стационарному}
\end{table}

\subsubsection{Нестационарное решение}
\[
  u(r, t) = r + t
\]

Тогда получаем значение f, как:
\[
  f(r, t) = 1 - \frac{\cos(r) - r\sin(r) + 6}{2r} + \frac{t\sin(r)+\sin(r)}{2} + 2t + 2r
\]

Значения $ \nu_1 $ и $ \nu_2 $ такие:
\[
  \nu_1 = \frac{\sqrt{3}}{4} + 3
\]
\[
  \nu_2 = 3.25
\]

После выполнения программы получаем следующие значения:
\begin{table}[H]
  \centering
  \begin{tabular}{c | c | c}
    \toprule
    H & $ \left\lVert \varepsilon \right\rVert  $, явный метод & $ \left\lVert \varepsilon \right\rVert  $, неявный метод \\
    \midrule
    1e-1 & 5.57e+20 & 3.18e-01\\
    1e-2 & 1.18e+139 & 3.3e-02\\
    1e-3 & 6.48e+300 & 3.31e-03\\
    1e-4 & 1.69e-05 & 3.57e-04\\
    1e-5 & 1.69e-05 & 1.29e-05\\
    1e-6 & 1.69e-05 & 8.48e-05\\
    \bottomrule
  \end{tabular}
  \caption{Погрешность теста ДУ с нестационарным решением}
\end{table}

Полученные данные дают нам повод убедиться, что явный метод Эйлера непременим для решения жестких систем.
При небольшом шаге можно заметить "взрыв погрешности". 

В свою же очередь неявный метод Эйлера требует решения системы линейных алгебраических уравнений, что требует
значительного увеличения числа вычислений.
	\newpage

	\section{Заключение}
	\subsection{Вывод}
	Задание выполнено полностью. Были написаны: метод для приближенного решения дифференциального уравнения, явный и неявный метод Эйлера.
	Все методы были протестированы. Были выявлены недостатки и преимущества явного и неявного метода Эйлера.
	\newpage
	\subsection{Код}
	\inputminted{c++}{../lab2/src/main.cpp}
	\inputminted{c++}{../lab2/src/data_table.hpp}
	\inputminted{c++}{../lab2/include/tma.hpp}
	\inputminted{c++}{../lab2/include/euler.hpp}
	\inputminted{c++}{../lab2/include/utils.hpp}
	\inputminted{c++}{../lab2/include/utils/balance_utils.hpp}
	\inputminted{c++}{../lab2/include/utils/data.hpp}
	\inputminted{c++}{../lab2/include/utils/grid.hpp}
\end{document}