\subsection{Разностная схема}
Введем основную сетку, где N - число разбиений.
\[ 
  r_0 < r_1 < \dots < r_N,\ r_i \in [R_L, R_R],\ r_0 = R_L,\ r_N = R_R
\]
\[
  h_i =r_i - r_{i-1},\ i=1,2, \dots, N
\]
\[
  r_{r-0.5} = \frac{r_i - r_{i-1}}{2},\ i=1,2, \dots, N
\]
Введем дополнительную сетку:
\[
  \hbar_i = \begin{cases}
    \frac{h_i + 1}{2},\ i = 0 \\ \\
    \frac{h_i + h_{i+1}}{2},\ i = 1, 2, \dots, N-1 \\ \\
    \frac{h_i}{2},\ i = N
  \end{cases}
\]
Напишем наше уравнение:
\begin{align*}
  \frac{\partial u}{\partial t} = \frac{1}{r} \frac{\partial}{\partial r}
  \left ( rk(r, t)\frac{\partial u}{\partial r} \right ) - q(r, t)u + f(r,t)
\end{align*}

\begin{align*}
  r \cdot \frac{\partial u}{\partial t} = \frac{\partial}{\partial r}
  \left ( rk\frac{\partial u}{\partial r} \right ) - r \cdot qu + r \cdot f
\end{align*}

\[
  \int\limits^{x_i+\frac{1}{2}}_{x_i-\frac{1}{2}} r \frac{\partial u}{\partial t} dt =
  \int\limits^{x_i+\frac{1}{2}}_{x_i-\frac{1}{2}} \frac{\partial}{\partial r} \left ( rk\frac{\partial u}{\partial r} \right ) dr
  - \int\limits^{x_i+\frac{1}{2}}_{x_i-\frac{1}{2}} rqu dr - \int\limits^{x_i+\frac{1}{2}}_{x_i-\frac{1}{2}} rf dr
\]

\[
  \int\limits^{x_i+\frac{1}{2}}_{x_i-\frac{1}{2}} \varphi(x)dx \approx \hbar_i \varphi(x)
\]
\[
  \hbar_i r_i \frac{d v_i}{dt} = \left . rk \frac{\partial u}{\partial r} \right \vert_{r=r_{i+\frac{1}{2}}}
  - \left . rk \frac{\partial u}{\partial r} \right \vert_{r=r_{i-\frac{1}{2}}} -\hbar_i r_i q v_i + \hbar_i r_i f_i
\]

\begin{align*}
  &\left . rk \frac{\partial u}{\partial r} \right \vert_{r=r_{i+\frac{1}{2}}} = r_{i + \frac{1}{2}} k_{i + \frac{1}{2}} \frac{v_{i+1}-v_i}{h_{i + 1}} \\
  &\left . rk \frac{\partial u}{\partial r} \right \vert_{r=r_{i-\frac{1}{2}}} = = r_{i - \frac{1}{2}} k_{i - \frac{1}{2}} \frac{v_{i}-v_{i-1}}{h_{i}}
\end{align*}

Тем самым мы получили нашу разностную схему внутри промежутка:
\[
  \hbar_i r_i \frac{d v_i}{dt} = r_{i + \frac{1}{2}} k_{i + \frac{1}{2}} \frac{v_{i+1}-v_i}{h_{i + 1}} 
  - r_{i - \frac{1}{2}} k_{i - \frac{1}{2}} \frac{v_{i}-v_{i-1}}{h_{i}} -\hbar_i r_i q v_i + \hbar_i r_i f_i,\ i = 1, 2, \dots, N-1
\]
\[
  \frac{d v_i}{dt} = \frac{r_{i + \frac{1}{2}} k_{i + \frac{1}{2}}}{\hbar_i r_i h_{i + 1}} (v_{i+1}-v_i)
  - \frac{r_{i - \frac{1}{2}} k_{i - \frac{1}{2}}}{\hbar_i r_i h_{i}} (v_{i}-v_{i-1}) - q v_i + f_i,\ i = 1, 2, \dots, N-1
\]

Проведем аппроксимацию граничного условия слева:
\[
  \int\limits^{x_i+\frac{1}{2}}_{x_i} r \frac{\partial u}{\partial t} dt =
  \int\limits^{x_i+\frac{1}{2}}_{x_i} \frac{\partial}{\partial r} \left ( rk\frac{\partial u}{\partial r} \right ) dr
  - \int\limits^{x_i+\frac{1}{2}}_{x_i} rqu dr - \int\limits^{x_i+\frac{1}{2}}_{x_i} rf dr,\ i = 0
\]
\[
  \hbar_i r_i \frac{d v_i}{dt} = \left . rk \frac{\partial u}{\partial r} \right \vert_{r=r_{i+\frac{1}{2}}}
  - \left . rk \frac{\partial u}{\partial r} \right \vert_{r=r_i} -\hbar_i r_i q v_i + \hbar_i r_i f_i,\ i = 0
\]

Граничное условие
\[
  k \left. \frac{\partial u}{\partial r}\right\vert_{r = R_L} = -\nu_1(t)
\]

Получаем:
\[
  \hbar_i r_i \frac{d v_i}{dt} = r_{i + \frac{1}{2}} k_{i + \frac{1}{2}} \frac{v_{i+1}-v_i}{h_{i + 1}} 
  - r_i \cdot \nu_1 - \hbar_i r_i q v_i + \hbar_i r_i f_ix-\frac{1}{2},\ i = 0
\]
\[
  \frac{d v_i}{dt} = \frac{r_{i + \frac{1}{2}} k_{i + \frac{1}{2}}}{\hbar_i r_i h_{i + 1}} (v_{i+1}-v_i)
  - \frac{\nu_1}{\hbar_i} - q v_i + f_i,\ i = 0
\]

Проведем аппроксимацию граничного условия справа:
\[
  \int\limits^{x_i}_{x_i-\frac{1}{2}} r \frac{\partial u}{\partial t} dt =
  \int\limits^{x_i}_{x_i-\frac{1}{2}} \frac{\partial}{\partial r} \left ( rk\frac{\partial u}{\partial r} \right ) dr
  - \int\limits^{x_i}_{x_i-\frac{1}{2}} rqu dr - \int\limits^{x_i}_{x_i-\frac{1}{2}} rf dr,\ i = N
\]
\[
  \hbar_i r_i \frac{d v_i}{dt} = \left . rk \frac{\partial u}{\partial r} \right \vert_{r=r_i}
  - \left . rk \frac{\partial u}{\partial r} \right \vert_{r=r_{i-\frac{1}{2}}} -\hbar_i r_i q v_i + \hbar_i r_i f_i,\ i = N
\]

Граничное условие:
\[
  -k \left. \frac{\partial u}{\partial r}\right\vert_{r = R_R} = -\nu_2(t)
\]

Получаем:
\[
  \hbar_i r_i \frac{d v_i}{dt} = r_{i } \frac{\nu_2}{h_{i + 1}} 
  - r_{i - \frac{1}{2}} k_{i - \frac{1}{2}} \frac{v_{i}-v_{i-1}}{h_{i}} -\hbar_i r_i q v_i + \hbar_i r_i f_i,\ i = N
\]
\[
  \frac{d v_i}{dt} = \frac{\nu_2}{\hbar_i}
  - \frac{r_{i - \frac{1}{2}} k_{i - \frac{1}{2}}}{\hbar_i r_i h_{i}} (v_{i}-v_{i-1}) - q v_i + f_i,\ i = N
\]

Введем обозначения:
\[
  B_1 =\frac{r_{i + \frac{1}{2}} k_{i + \frac{1}{2}}}{\hbar_i r_i h_{i + 1}} 
  \quad B_2 = \frac{r_{i - \frac{1}{2}} k_{i - \frac{1}{2}}}{\hbar_i r_i h_{i}}
\]

Запишем полученные уравнения:
\[
  \begin{cases}
    \frac{d v_i}{dt} = B_1 (v_{i+1}-v_i)
    - \frac{\nu_1}{\hbar_i} - q v_i + f_i,\ i = 0 \\

    \frac{d v_i}{dt} = B_1 (v_{i+1}-v_i)
    - B_2 (v_{i}-v_{i-1}) - q v_i + f_i,\ i = 1, 2, \dots, N-1 \\

    \frac{d v_i}{dt} = \frac{\nu_2}{\hbar_i}
  - B_2 (v_{i}-v_{i-1}) - q v_i + f_i,\ i = N
  \end{cases}
\]

Сгруппируем значения:
\[
  \begin{cases}
    \frac{d v_i}{dt} = B_1 v_{i+1} - (B_1 + q_i) v_i - \frac{\nu_1}{\hbar_i} + f_i,\ i=0 \\
    \frac{d v_i}{dt} = B_1 v_{i+1} - (B_1 + B_2 + q_i) v_i - B_2 v_{i - 1} + f_i,\ i = 1, 2,\dots, N-1\\
    \frac{d v_i}{dt} = -(B_2 + q_i)v_i - B_2 v_{i - 1} + \frac{\nu_2}{\hbar_i} + f_i,\ i = N
  \end{cases}
\]

Мы получили систему:
\begin{align*}
  \begin{pmatrix}
    \frac{d v_0}{dt} \\
    \frac{d v_1}{dt} \\
    \vdots \\
    \vdots \\
    \vdots \\
    \frac{d v_{n-1}}{dt} \\
    \frac{d v_{n}}{dt} 
  \end{pmatrix} 
  =
  \begin{pmatrix}
    c_0 & b_0 & &  &  &  & 0 \\
    a_1 & c_1 & b_1 &  & & & \\
     &   \ddots & \ddots & \ddots & & & \\
     & &  \ddots & \ddots & \ddots & &  \\
     & & & \ddots & \ddots & \ddots &  \\
     &  &  & & a_{n-1} & c_{n-1} & b_{n-1} \\
     0 & & & &  &a_n & c_n
  \end{pmatrix}
  \begin{pmatrix}
    x_0 \\
    x_1 \\
    \vdots \\
    \vdots \\
    \vdots \\
    x_{n-1} \\
    x_n
  \end{pmatrix} +
  \begin{pmatrix}
    g_0 \\
    g_1 \\
    \vdots \\
    \vdots \\
    \vdots \\
    g_{n-1} \\
    g_n
  \end{pmatrix}
\end{align*}

где:

\begin{align*}
  & & &c_0=-(B_1 + q_0) & &b_0=B_1 & &g_0= \frac{\nu_1}{\hbar_0} + f_0 \\
  &a_i=B_2 & &c_i=-(B_1 + B_2 + q_i) & &b_i=B_1 & &g_i=f_i \\
  &a_N=B_2 & &c_N=-(B_2 + q_N) & & & &g_N= \frac{\nu_2}{\hbar_N} + f_N
\end{align*}

\subsection{Решение системы ОДУ}

Система имеет вид:
\[
  \frac{d v_i}{dt} = Av + g
\]

Введем дискретизацию по времени и проинтегрируем на одном из промежутков:
\[
  \int\limits_{t_{n}}^{t_n+1} \frac{d v_i}{dt} \,dt = \int\limits_{t_{n}}^{t_n+1} (Av + g) dt
\]
\[
  v(t_{n+1}) - v(t_n) = \int\limits_{t_{n}}^{t_n+1} (Av + g) dt
\]
\[
  v(t_{n+1}) = v(t_n) + \int\limits_{t_{n}}^{t_n+1} (Av + g) dt
\]

Добавим к полученному уравнению начальное условие и тем самым получаем систему:
\[
  \begin{cases}
    v(t_{n+1}) = v(t_n) + \int\limits_{t_{n}}^{t_n+1} (Av + g) dt \\
    v(t_0) = \varphi(r)
  \end{cases}
\]

Решение задачи сводится к поиску значения интеграла.