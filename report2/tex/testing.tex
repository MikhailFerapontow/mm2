\subsection{Тестирование}

Погрешность решения находится следующим образом:
\[
  \max_{t_0, \cdots, t_n} = \left\lVert \varepsilon(t) \right\rVert 
\]
\[
  \left\lVert \varepsilon(t) \right\rVert  = \left\lVert \tilde{v(t)} - v(t) \right\rVert 
\]

Для всех тестов возьмем одинаковое интервал для r и t.
\[
  r \in [\frac{\pi}{6}, \frac{\pi}{3}]
\]
\[
  T \in [0, 1]
\]

Для всех тестов возьмем одинаковые значения k и q:
\[
  k = \frac{\cos(r)}{2} + 3
\]
\[
  q = \frac{\sin(r)}{2} + 2
\]

Также для всех тестов возьмем одинаковое число разбиений:
\[
  N = 16
\]

\subsubsection{Стационарное решение}
\[
  u(r, t) = r
\]

Тогда получаем значение f, как:
\[
  f(r, t) = \frac{-\cos(r) + r \sin(r)-6 + r^2 \sin(r) + 4r^2}{2r}
\]

Значения $ \nu_1 $ и $ \nu_2 $ такие:
\[
  \nu_1 = \frac{\sqrt{3}}{4} + 3
\]
\[
  \nu_2 = 3.25
\]

После выполнения программы получаем следующие значения:
\begin{table}[H]
  \centering
  \begin{tabular}{c | c | c}
    \toprule
    H & $ \left\lVert \varepsilon \right\rVert  $, явный метод & $ \left\lVert \varepsilon \right\rVert  $, неявный метод \\
    \midrule
    1e-1 & 2.25e+22 & 5.05e-5\\
    1e-2 & 2.13e+140 & 5.05e-5\\
    1e-3 & 1.09e+330 & 5.05e-5\\
    1e-4 & 3.57e-10 & 5.05e-5\\
    1e-5 & 3.88e-9 & 5.05e-5\\
    1e-6 & 3.84e-8 & 5.05e-5\\
    \bottomrule
  \end{tabular}
  \caption{Погрешность теста ДУ со стационарномым решением}
\end{table}

\subsubsection{Решение стремящееся к стационарному}
\[
  u(r, t) = 4 + e^{-10t}
\]

Тогда получаем значение f, как:
\[
  f(r, t) = \frac{-16 + 4e^{10t}\sin(r) + \sin(r) + 16e^{10t}}{2e^{10t}}
\]

Значения $ \nu_1 $ и $ \nu_2 $ такие:
\[
  \nu_1 = 0
\]
\[
  \nu_2 = 0
\]

После выполнения программы получаем следующие значения:
\begin{table}[H]
  \centering
  \begin{tabular}{c | c | c}
    \toprule
    H & $ \left\lVert \varepsilon \right\rVert  $, явный метод & $ \left\lVert \varepsilon \right\rVert  $, неявный метод \\
    \midrule
    1e-1 & 2.96e+14 & 1.65e-01\\
    1e-2 & 4.55e+140 & 1.36e-02\\
    1e-3 & 4.52e+292 & 2.94e-03\\
    1e-4 & 2.76e-05 & 4.54e-04\\
    1e-5 & 2.76e-05 & 2.98e-04\\
    1e-6 & 2.76e-03 & 2.44e-04\\
    \bottomrule
  \end{tabular}
  \caption{Погрешность теста ДУ с решением стремящимся к стационарному}
\end{table}

\subsubsection{Нестационарное решение}
\[
  u(r, t) = r + t
\]

Тогда получаем значение f, как:
\[
  f(r, t) = 1 - \frac{\cos(r) - r\sin(r) + 6}{2r} + \frac{t\sin(r)+\sin(r)}{2} + 2t + 2r
\]

Значения $ \nu_1 $ и $ \nu_2 $ такие:
\[
  \nu_1 = \frac{\sqrt{3}}{4} + 3
\]
\[
  \nu_2 = 3.25
\]

После выполнения программы получаем следующие значения:
\begin{table}[H]
  \centering
  \begin{tabular}{c | c | c}
    \toprule
    H & $ \left\lVert \varepsilon \right\rVert  $, явный метод & $ \left\lVert \varepsilon \right\rVert  $, неявный метод \\
    \midrule
    1e-1 & 5.57e+20 & 3.18e-01\\
    1e-2 & 1.18e+139 & 3.3e-02\\
    1e-3 & 6.48e+300 & 3.31e-03\\
    1e-4 & 1.69e-05 & 3.57e-04\\
    1e-5 & 1.69e-05 & 1.29e-05\\
    1e-6 & 1.69e-05 & 8.48e-05\\
    \bottomrule
  \end{tabular}
  \caption{Погрешность теста ДУ с нестационарным решением}
\end{table}

Полученные данные дают нам повод убедиться, что явный метод Эйлера непременим для решения жестких систем.
При небольшом шаге можно заметить "взрыв погрешности". 

В свою же очередь неявный метод Эйлера требует решения системы линейных алгебраических уравнений, что требует
значительного увеличения числа вычислений.