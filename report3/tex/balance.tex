\section{Разностная схема}
Введем основную сетку:
\begin{align*}
  &N_r - \text{число разбиений на} [0, R] & &N_z - \text{число разбиений на} [0, L] \\
  &r_0 < r_1 < \cdots < r_N & &z_0 < z_1 < \cdots < z_N \\
  &r_0 = 0,\quad r_N = R & &z_0 = 0,\quad z_N = L \\
  &h_r = \frac{R - 0}{N_r} & &h_z = \frac{L - 0}{N_z}
\end{align*}

Введем дополнительную сетку:
\begin{align*}
  &r_{i-\frac{1}{2}} = \frac{r_i + r_{i - 1}}{2}\quad i=1,\cdots, N_r & &z_{j-\frac{1}{2}} = \frac{z_j + z_{j - 1}}{2}\quad j=1,\cdots, N_z \\
  & \hbar_i = \begin{cases}
    \frac{h_r}{2},\ i = 0 \\ \\
    h_r,\ i = 1, 2, \dots, N_r-1 \\ \\
    \frac{h_r}{2},\ i = N_r
  \end{cases} &
  & \hbar_j = \begin{cases}
    \frac{h_z}{2},\ j = 0 \\ \\
    h_z,\ j = 1, 2, \dots, N_z-1 \\ \\
    \frac{h_z}{2},\ j = N_z
  \end{cases}
\end{align*}

Преобразуем наше начальное уравнение домножив на r

\[
  - \left [ \pdv{}{r} \left ( r k_1(r, z) \pdv{u}{r} \right ) 
  + \pdv{}{z} \left ( rk_2(r, z)\pdv{u}{z} \right ) \right ] = rf(r, z)
\]

\subsection{Внутренние точки}
Проинтегрируем данное нам в задании уравнение во внутренних точках сетки:

\[
  - \mLim{+\frac{1}{2}}{-\frac{1}{2}}{+\frac{1}{2}}{-\frac{1}{2}} \left [ \pdv{}{r} \left ( r k_1(r, z) \pdv{u}{r} \right ) 
  + \pdv{}{z} \left ( rk_2(r, z)\pdv{u}{z} \right ) \right ] dr dz = \mLim{+\frac{1}{2}}{-\frac{1}{2}}{+\frac{1}{2}}{-\frac{1}{2}} rf(r, z) dr dz
\]

Получим:

\begin{align*}
  &- \left [
   \mLimZ{z}{+\frac{1}{2}}{-\frac{1}{2}}{j}  \left . r k_1(r, z) \pdv{u}{r} \right \vert_{r = r_{i + \frac{1}{2}}} dz
  - \mLimZ{z}{+\frac{1}{2}}{-\frac{1}{2}}{j} \left . r k_1(r, z) \pdv{u}{r} \right \vert_{r = r_{i - \frac{1}{2}}} dz
  \right . \\
  &\left . + \mLimS{r}{+\frac{1}{2}}{-\frac{1}{2}} \left . rk_2(r, z)\pdv{u}{z} \right \vert_{z = z_{j + \frac{1}{2}}} dr
  - \mLimS{r}{+\frac{1}{2}}{-\frac{1}{2}} \left . rk_2(r, z)\pdv{u}{z} \right \vert_{z = z_{j - \frac{1}{2}}} dr
  \right ] = \mLim{+\frac{1}{2}}{-\frac{1}{2}}{+\frac{1}{2}}{-\frac{1}{2}} rf(r, z) dr dz
\end{align*}

Воспользуемся формулами численного дифференцирования:
\[
  \left . k_1(r, z) \pdv{u}{r} \right \vert_{r = r_{i - \frac{1}{2}}}
  \approx k_1(r_{i - \frac{1}{2}}, z) 
  \frac{v_{i, j} - v_{i - 1, j}}{h_r}
\]

\[
  \left . k_2(r, z) \pdv{u}{r} \right \vert_{z = z_{j - \frac{1}{2}}}
  \approx k_2(r, z_{j - \frac{1}{2}}) 
  \frac{v_{i, j} - v_{i, j - 1}}{h_z}
\]

Также воспользуемся формулой средних прямоугольников:
\[
  \mLimS{r}{+\frac{1}{2}}{-\frac{1}{2}} r \varphi(r, z) dr
  = \hbar_i r_i \varphi_i
\]
\[
  \mLim{+\frac{1}{2}}{-\frac{1}{2}}{+\frac{1}{2}}{-\frac{1}{2}} r \varphi(r, z) drdz
  = \hbar_i\hbar_j r_i \varphi_{i, j}
\]

В итоге получим уравнение разностной схемы в общем виде во внутренних точках сетки, а то есть для
$ i = 1, 2, \dots, N_r - 1 $ и $ j = 1, 2, \dots, N_z - 1$.
\begin{align*}
  &- \left [ 
  \hbar_j r_{i+\frac{1}{2}} k_1(r_{i+\frac{1}{2}}, z_j) \frac{v_{i+1, j} - v_{i, j}}{h_{r}}
  - \hbar_j r_{i-\frac{1}{2}} k_1(r_{i-\frac{1}{2}}, z_j) \frac{v_{i, j} - v_{i - 1, j}}{h_{r}}
  \right . \\
  &\left .
  + \hbar_i r_{i} k_2(r_i, z_{j+\frac{1}{2}}) \frac{v_{i, j + 1} - v_{i, j}}{h_{z}}
  - \hbar_i r_{i} k_2(r_i, z_{j-\frac{1}{2}}) \frac{v_{i, j} - v_{i, j - 1}}{h_z}
  \right ]  = \hbar_i \hbar_j r_i f_{i, j}
\end{align*}

Так как мы имеем равномерно разбитую сетку, мы можем перейти к обозначениям для основной сетки для $ \hbar_i $ и $ \hbar_j $:

\begin{align*}
  &- \left [ 
  h_z r_{i+\frac{1}{2}} k_1(r_{i+\frac{1}{2}}, z_j) \frac{v_{i+1, j} - v_{i, j}}{h_{r}}
  - h_z r_{i-\frac{1}{2}} k_1(r_{i-\frac{1}{2}}, z_j) \frac{v_{i, j} - v_{i - 1, j}}{h_{r}}
  \right . \\
  &\left .
  + h_r r_{i} k_2(r_i, z_{j+\frac{1}{2}}) \frac{v_{i, j + 1} - v_{i, j}}{h_{z}}
  - h_r r_{i} k_2(r_i, z_{j-\frac{1}{2}}) \frac{v_{i, j} - v_{i, j - 1}}{h_z}
  \right ]  = h_r h_z r_i f_{i, j}
\end{align*}

\subsection{На левой границе}

Проинтегрируем данное в задании уравнение на левой границе сетки.

\[
  - \mLim{+\frac{1}{2}}{}{+\frac{1}{2}}{-\frac{1}{2}} \left [ \pdv{}{r} \left ( r k_1(r, z) \pdv{u}{r} \right ) 
  + \pdv{}{z} \left ( rk_2(r, z)\pdv{u}{z} \right ) \right ] dr dz = \mLim{+\frac{1}{2}}{}{+\frac{1}{2}}{-\frac{1}{2}} rf(r, z) dr dz
\]

Получаем:
\begin{align*}
  &- \left [
   \mLimZ{z}{+\frac{1}{2}}{-\frac{1}{2}}{j}  \left . r k_1(r, z) \pdv{u}{r} \right \vert_{r = r_{i + \frac{1}{2}}} dz
  - \mLimZ{z}{+\frac{1}{2}}{-\frac{1}{2}}{j} \left . r k_1(r, z) \pdv{u}{r} \right \vert_{r = r_{i}} dz
  \right . \\
  &\left . + \mLimS{r}{+\frac{1}{2}}{} \left . rk_2(r, z)\pdv{u}{z} \right \vert_{z = z_{j + \frac{1}{2}}} dr
  - \mLimS{r}{+\frac{1}{2}}{} \left . rk_2(r, z)\pdv{u}{z} \right \vert_{z = z_{j - \frac{1}{2}}} dr
  \right ] = \mLim{+\frac{1}{2}}{}{+\frac{1}{2}}{-\frac{1}{2}} rf(r, z) dr dz
\end{align*}

Имеем граничное условие:
\[
  \left . u \right \vert_{r=0} - \text{ограничено, т. е } \left . \frac{\partial u}{ \partial r} \right |_{r = 0} = 0
\]

\[
  \mLimS{r}{+ \frac{1}{2}}{} rf dr \approx f_i \mLimS{r}{ + \frac{1}{2}}{} r dr = 
  f_i \frac{r^2_{i + \frac{1}{2}}}{2} = \frac{h_r}{2} f_i \frac{r_{i + \frac{1}{2}}}{2},
  \quad i = 0, \quad r_i = 0, r_{i + \frac{1}{2}} = \frac{h_r}{2}
\]

Получаем уравнение разностной схемы в общем виде на левой границе сетки, а то есть
для $ i = 0 $ и $ j = 1,2, \dots, N_z-1 $.

\begin{align*}
  &- \left [ 
  \hbar_j r_{i+\frac{1}{2}} k_1(r_{i+\frac{1}{2}}, z_j) \frac{v_{i+1, j} - v_{i, j}}{h_{i + 1}}
  - 0
  \right . \\
  &\left .
  + \hbar_i \frac{r_{i + \frac{1}{2}}}{2} k_2(r_i, z_{j+\frac{1}{2}}) \frac{v_{i, j + 1} - v_{i, j}}{h_{j + 1}}
  - \hbar_i \frac{r_{i + \frac{1}{2}}}{2} k_2(r_i, z_{j-\frac{1}{2}}) \frac{v_{i, j} - v_{i, j - 1}}{h_j}
  \right ]  = \hbar_i \hbar_j \frac{r_{i + \frac{1}{2}}}{2} f_{i, j}
\end{align*}

Перейдём к обозначениям основной сетки для $ \hbar_i $ и $ \hbar_j $:

\begin{align*}
  &- \left [ 
    h_z r_{\frac{1}{2}} k_1(r_{\frac{1}{2}}, z_j) \frac{v_{1, j} - v_{0, j}}{h_{r}}
    - 0
    \right . \\
    &\left .
    + \frac{h_r}{2} \frac{r_{\frac{1}{2}}}{2} k_2(r_0, z_{j+\frac{1}{2}}) \frac{v_{0, j + 1} - v_{0, j}}{h_{z}}
    - \frac{h_r}{2} \frac{r_{\frac{1}{2}}}{2} k_2(r_0, z_{j-\frac{1}{2}}) \frac{v_{0, j} - v_{0, j - 1}}{h_z}
    \right ]  = \frac{h_r}{2} \frac{r_{\frac{1}{2}}}{2} h_z f_{0, j}
\end{align*}

\subsection{На правой границе}
Проинтегрируем данное в задании уравнение на правой границе сетки.

\[
  - \mLim{}{-\frac{1}{2}}{+\frac{1}{2}}{-\frac{1}{2}} \left [ \pdv{}{r} \left ( r k_1(r, z) \pdv{u}{r} \right ) 
  + \pdv{}{z} \left ( rk_2(r, z)\pdv{u}{z} \right ) \right ]dr dz = \mLim{}{-\frac{1}{2}}{+\frac{1}{2}}{-\frac{1}{2}} rf(r, z)dr dz
\]

Получаем:
\begin{align*}
  &- \left [
   \mLimZ{z}{+\frac{1}{2}}{-\frac{1}{2}}{j}  \left . r k_1(r, z) \pdv{u}{r} \right \vert_{r = r_{i}} dz
  - \mLimZ{z}{+\frac{1}{2}}{-\frac{1}{2}}{j} \left . r k_1(r, z) \pdv{u}{r} \right \vert_{r = r_{i - \frac{1}{2}}} dz
  \right . \\
  &\left . + \mLimS{r}{}{-\frac{1}{2}} \left . rk_2(r, z)\pdv{u}{z} \right \vert_{z = z_{j + \frac{1}{2}}} dr
  - \mLimS{r}{}{-\frac{1}{2}} \left . rk_2(r, z)\pdv{u}{z} \right \vert_{z = z_{j - \frac{1}{2}}} dr
  \right ] = \mLim{}{-\frac{1}{2}}{+\frac{1}{2}}{-\frac{1}{2}} rf(r, z) dr dz
\end{align*}

Имеем граничное условие:
\[
  \left . -k_1 \pdv{u}{r} \right \vert_{r=R} = \chi_2 \left . u \right \vert_{r=R} - \varphi_2(z)
\]

Получаем уравнение разностной схемы в общем виде на правой границе сетки, а то есть
для $ i = N_r $ и $ j = 1,2, \dots, N_z-1 $.

\begin{align*}
  &- \left [ 
  -\hbar_j r_i ( \chi_2 v_i - \varphi_2(z))
  - \hbar_j r_{i-\frac{1}{2}} k_1(r_{i-\frac{1}{2}}, z_j) \frac{v_{i, j} - v_{i - 1, j}}{h_{r}}
  \right . \\
  &\left .
  + \hbar_i r_{i} k_2(r_i, z_{j+\frac{1}{2}}) \frac{v_{i, j + 1} - v_{i, j}}{h_{z}}
  - \hbar_i r_{i} k_2(r_i, z_{j-\frac{1}{2}}) \frac{v_{i, j} - v_{i, j - 1}}{h_z}
  \right ]  = \hbar_i \hbar_j r_i f_{i, j}
\end{align*}

Перейдём к обозначениям основной сетки для $ \hbar_i $ и $ \hbar_j $:

\begin{align*}
  &- \left [ 
  -h_z r_{N_r} ( \chi_2 v_{N_r} - \varphi_2(z))
  - h_z r_{N_r-\frac{1}{2}} k_1(r_{N_r-\frac{1}{2}}, z_j) \frac{v_{N_r, j} - v_{N_r - 1, j}}{h_{r}}
  \right . \\
  &\left .
  + \frac{h_r}{2} r_{N_r} k_2(r_{N_r}, z_{j+\frac{1}{2}}) \frac{v_{N_r, j + 1} - v_{N_r, j}}{h_{z}}
  - \frac{h_r}{2} r_{N_r} k_2(r_{N_r}, z_{j-\frac{1}{2}}) \frac{v_{N_r, j} - v_{N_r, j - 1}}{h_z}
  \right ]  = \frac{h_r}{2} r_{N_r} h_z f_{N_r, j}
\end{align*}

\subsection{На нижней границе}
Проинтегрируем данное в задании уравнение на нижней границе сетки.

\[
  - \mLim{+\frac{1}{2}}{-\frac{1}{2}}{+\frac{1}{2}}{} \left [ \pdv{}{r} \left ( r k_1(r, z) \pdv{u}{r} \right ) 
  + \pdv{}{z} \left ( rk_2(r, z)\pdv{u}{z} \right ) \right ] drdz = \mLim{+\frac{1}{2}}{-\frac{1}{2}}{+\frac{1}{2}}{} rf(r, z) dr dz
\]

Получаем:
\begin{align*}
  &- \left [
   \mLimZ{z}{+\frac{1}{2}}{}{j}  \left . r k_1(r, z) \pdv{u}{r} \right \vert_{r = r_{i + \frac{1}{2}}} dz
  - \mLimZ{z}{+\frac{1}{2}}{}{j} \left . r k_1(r, z) \pdv{u}{r} \right \vert_{r = r_{i - \frac{1}{2}}} dz
  \right . \\
  &\left . + \mLimS{r}{+\frac{1}{2}}{-\frac{1}{2}} \left . rk_2(r, z)\pdv{u}{z} \right \vert_{z = z_{j + \frac{1}{2}}} dr
  - \mLimS{r}{+\frac{1}{2}}{-\frac{1}{2}} \left . rk_2(r, z)\pdv{u}{z} \right \vert_{z = z_{j}} dr
  \right ] = \mLim{+\frac{1}{2}}{-\frac{1}{2}}{+\frac{1}{2}}{} rf(r, z) dr dz
\end{align*}

Имеем граничное условие:
\[
  \left . k_2 \pdv{u}{z} \right \vert_{z=0} = \chi_3 \left . u \right \vert_{z=0} - \varphi_3(r)
\]

Получаем уравнение разностной схемы в общем виде на нижней границе сетки, а то есть
для $ i = 1,2, \dots, N_r-1 $ и $ j = 0 $.

\begin{align*}
  - \left [ 
  \hbar_j r_{i+\frac{1}{2}} k_1(r_{i+\frac{1}{2}}, z_j) \frac{v_{i+1, j} - v_{i, j}}{h_{r}}
  - \hbar_j r_{i-\frac{1}{2}} k_1(r_{i-\frac{1}{2}}, z_j) \frac{v_{i, j} - v_{i - 1, j}}{h_{r}}
  \right . \\
  \left .
  + \hbar_i r_{i} k_2(r_i, z_{j+\frac{1}{2}}) \frac{v_{i, j + 1} - v_{i, j}}{h_{z}}
  - \hbar_i r_i (\chi_3 v_i - \varphi_3(r))
  \right ]  = \hbar_i \hbar_j r_i f_{i, j}
\end{align*}

Перейдём к обозначениям основной сетки для $ \hbar_i $ и $ \hbar_j $:

\begin{align*}
  - \left [ 
  \frac{h_z}{2} r_{i+\frac{1}{2}} k_1(r_{i+\frac{1}{2}}, z_0) \frac{v_{i+1, 0} - v_{i, 0}}{h_{r}}
  - \frac{h_z}{2} r_{i-\frac{1}{2}} k_1(r_{i-\frac{1}{2}}, z_0) \frac{v_{i, 0} - v_{i - 1, 0}}{h_{r}}
  \right . \\
  \left .
  + h_r r_{i} k_2(r_i, z_{\frac{1}{2}}) \frac{v_{i, 1} - v_{i, 0}}{h_{z}}
  - h_r r_i (\chi_3 v_{i, 0} - \varphi_3(r))
  \right ]  = h_r \frac{h_z}{2} r_i f_{i, 0}
\end{align*}

\subsection{На верхней границе}
В качестве уравнения разностной схемы на верхней границе сетки, 
а то есть $ i = 1, 2, \dots, N_r - 1 $ и $ j = N_z $, возьмём известное граничное условие:

\[
  \left . u \right \vert_{z=L} = \varphi_r(r) 
\]

\[
  v_{i,N_z} = \varphi(r_i)
\]

\subsection{Левый-нижний угол}

Проинтегрируем данное в задании уравнение в левой-нижней граничной точке сетки.
\[
  - \mLim{+\frac{1}{2}}{}{+\frac{1}{2}}{} \left [ \pdv{}{r} \left ( r k_1(r, z) \pdv{u}{r} \right ) 
  + \pdv{}{z} \left ( rk_2(r, z)\pdv{u}{z} \right ) \right ] dr dz = \mLim{+\frac{1}{2}}{}{+\frac{1}{2}}{} rf(r, z) dr dz
\]

Получаем:
\begin{align*}
  &- \left [
   \mLimZ{z}{+\frac{1}{2}}{}{j}  \left . r k_1(r, z) \pdv{u}{r} \right \vert_{r = r_{i + \frac{1}{2}}} dz
  - \mLimZ{z}{+\frac{1}{2}}{}{j} \left . r k_1(r, z) \pdv{u}{r} \right \vert_{r = r_{i}} dz
  \right . \\
  &\left . + \mLimS{r}{+\frac{1}{2}}{} \left . rk_2(r, z)\pdv{u}{z} \right \vert_{z = z_{j + \frac{1}{2}}} dr
  - \mLimS{r}{+\frac{1}{2}}{} \left . rk_2(r, z)\pdv{u}{z} \right \vert_{z = z_{j}} dr
  \right ] = \mLim{+\frac{1}{2}}{}{+\frac{1}{2}}{} rf(r, z) dr dz
\end{align*}

Имеем граничное условие:
\[
  \left . u \right \vert_{r=0} - \text{ограничено, т. е } \left . \frac{\partial u}{ \partial r} \right |_{r = 0} = 0
\]

\[
  \mLimS{r}{+ \frac{1}{2}}{} rf dr \approx f_i \mLimS{r}{ + \frac{1}{2}}{} r dr = 
  f_i \frac{r^2_{i + \frac{1}{2}}}{2} = h_r f_i \frac{r_{i + \frac{1}{2}}}{2},
  \quad i = 0, \quad r_i = 0, r_{i + \frac{1}{2}} = \frac{h_r}{2}
\]

Также:
\[
  \left . k_2 \pdv{u}{z} \right \vert_{z=0} = \chi_3 \left . u \right \vert_{z=0} - \varphi_3(r) 
\]

Получаем уравнение разностной схемы в общем виде в левой-нижней граничной точке сетки, а то есть
для $ i = 0 $ и $ j = 0 $.

\begin{align*}
  &- \left [ 
  \hbar_j r_{i+\frac{1}{2}} k_1(r_{i+\frac{1}{2}}, z_j) \frac{v_{i+1, j} - v_{i, j}}{h_{r}}
  - 0
  \right . \\
  &\left .
  + \hbar_i \frac{r_{i + \frac{1}{2}}}{2} k_2(r_i, z_{j+\frac{1}{2}}) \frac{v_{i, j + 1} - v_{i, j}}{h_{z}}
  - \hbar_i \frac{r_{i + \frac{1}{2}}}{2} (\chi_3 \left . u \right \vert_{z=0} - \varphi_3(r_0))
  \right ]  = \hbar_i \hbar_j \frac{r_{i + \frac{1}{2}}}{2} f_{i, j}
\end{align*}

Перейдём к обозначениям основной сетки для $ \hbar_i $ и $ \hbar_j $:

\begin{align*}
  &- \left [ 
  \frac{h_z}{2} r_{\frac{1}{2}} k_1(r_{\frac{1}{2}}, z_0) \frac{v_{1, 0} - v_{0, 0}}{h_{r}}
  - 0
  \right . \\
  &\left .
  + \frac{h_r}{2} \frac{r_{\frac{1}{2}}}{2} k_2(r_0, z_{\frac{1}{2}}) \frac{v_{0, 1} - v_{0, 0}}{h_{z}}
  - \frac{h_r}{2} \frac{r_{\frac{1}{2}}}{2} (\chi_3 v_{0, 0} - \varphi_3(r_0))
  \right ]  = \frac{h_r}{2} \frac{h_z}{2} \frac{r_{\frac{1}{2}}}{2} f_{0, 0}
\end{align*}

\subsection{Левый-верхний угол}

В качестве уравнения разностной схемы в левой-верхней граничной точке сетки, 
а то есть $ i = 0 $ и $ j = N_z $, возьмём известное граничное условие:

\[
  \left . u \right \vert_{z=L} = \varphi_r(r) 
\]

\[
  v_{0,N_z} = \varphi(r_0)
\]

\subsection{Правый-верхний угол}

В качестве уравнения разностной схемы в левой-верхней граничной точке сетки, 
а то есть $ i = N_r $ и $ j = N_z $, возьмём известное граничное условие:

\[
  \left . u \right \vert_{z=L} = \varphi_r(r) 
\]

\[
  v_{N_r,N_z} = \varphi(r_{N_r})
\]

\subsection{Правый-нижний угол}

Проинтегрируем данное в задании уравнение в правой-нижней граничной точке сетки.
\[
  - \mLim{}{-\frac{1}{2}}{+\frac{1}{2}}{} \left [ \pdv{}{r} \left ( r k_1(r, z) \pdv{u}{r} \right ) 
  + \pdv{}{z} \left ( rk_2(r, z)\pdv{u}{z} \right ) \right ] = \mLim{}{-\frac{1}{2}}{+\frac{1}{2}}{} rf(r, z)
\]

Получаем:

\begin{align*}
  &- \left [
   \mLimZ{z}{+\frac{1}{2}}{}{j}  \left . r k_1(r, z) \pdv{u}{r} \right \vert_{r = r_i} dz
  - \mLimZ{z}{+\frac{1}{2}}{}{j} \left . r k_1(r, z) \pdv{u}{r} \right \vert_{r = r_{i - \frac{1}{2}}} dz
  \right . \\
  &\left . + \mLimS{r}{}{-\frac{1}{2}} \left . rk_2(r, z)\pdv{u}{z} \right \vert_{z = z_{j + \frac{1}{2}}} dr
  - \mLimS{r}{}{-\frac{1}{2}} \left . rk_2(r, z)\pdv{u}{z} \right \vert_{z = z_{j}} dr
  \right ] = \mLim{}{-\frac{1}{2}}{+\frac{1}{2}}{} rf(r, z) dr dz
\end{align*}

Имеем граничные условия:

\begin{align*}
  &\left . -k_1 \pdv{u}{r} \right \vert_{r=R} = \chi_2 \left . u \right \vert_{r=R} - \varphi_2(z) \\
  &\left . k_2 \pdv{u}{z} \right \vert_{z=0} = \chi_3 \left . u \right \vert_{z=0} - \varphi_3(r) 
\end{align*}

Получаем уравнение разностной схемы в общем виде в правой-нижней граничной точке сетки, а то есть
для $ i = N_r $ и $ j = 0 $.

\begin{align*}
  - \left [ 
  -\hbar_j r_i (\chi_2 \left . u \right \vert_{r=R} - \varphi_2(z) )
  - \hbar_j r_{i-\frac{1}{2}} k_1(r_{i-\frac{1}{2}}, z_j) \frac{v_{i, j} - v_{i - 1, j}}{h_{r}}
  \right . \\
  \left .
  + \hbar_i r_{i} k_2(r_i, z_{j+\frac{1}{2}}) \frac{v_{i, j + 1} - v_{i, j}}{h_{z}}
  - \hbar_i r_i (\chi_3 v_i - \varphi_3(r))
  \right ]  = \hbar_i \hbar_j r_i f_{i, j}
\end{align*}

Перейдём к обозначениям основной сетки для $ \hbar_i $ и $ \hbar_j $:

\begin{align*}
  - \left [ 
  -\frac{h_z}{2} r_{N_r} (\chi_2 v_{N_r, 0} - \varphi_2(z) )
  - \frac{h_z}{2} r_{N_r-\frac{1}{2}} k_1(r_{N_r-\frac{1}{2}}, z_0) \frac{v_{N_r, 0} - v_{N_r - 1, 0}}{h_{r}}
  \right . \\
  \left .
  + \frac{h_r}{2} r_{N_r} k_2(r_{N_r}, z_{\frac{1}{2}}) \frac{v_{N_r, 1} - v_{N_r, 0}}{h_{z}}
  - \frac{h_r}{2} r_{N_r} (\chi_3 v_{N_r, 0} - \varphi_3(r))
  \right ]  = \frac{h_r}{2} \frac{h_z}{2} r_{N_r} f_{N_r, 0}
\end{align*}