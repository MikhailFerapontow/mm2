\section{Тестирование}

Протестируем постренную модель на следующих тестовых наборах:

\begin{center}
  \begin{tabular}{*4{c}}
  \toprule
  \makecell{№ \\ теста} & $k_1(r, z)$ & $k_2(r, z)$ & $ u(r, z)$ \\
  \midrule
  1 & $ r + 2z $ & $ 3r + 4z $ & $1$ \\
  2 & $ (r + 2z)^2 $ & $ (3r + 4z)^2 $ & $1$ \\
  3 & $ (r + 2z)^2 $ & $ (3r + 4z)^2 $ & $r^2(z + 1)$ \\
  4 & $ (r + 2z)^3 $ & $ (3r + 4z)^3 $ & $r^2(z^2 + z + 1)$ \\
  \bottomrule
  \end{tabular}
\end{center}

Все тесты имеют общие значения:

\[
  r \in [0, 1],\quad z \in [0, 2]
\]

\[
  \chi_2 = 2,\quad \chi_3 = 3
\]

Функция $f$, $ \varphi_2 $, $ \varphi_3 $, $\varphi_4$ вычисляют программно
с использованием символичных вычислений.

Погрешность решения задачи вычисляется следующим образом:
\[
\delta_1 = \frac{\left|| v - \Tilde{v} \right||_1 }{\left|| v \right||_1 }
\]

\[
\delta_2 = \frac{\left|| v - \Tilde{v} \right||_2 }{\left|| v \right||_2 }
\]

\[
\delta_3 = \frac{\left|| v - \Tilde{v} \right||_\infty }{\left|| v \right||_\infty }
\]

Значение $ \varepsilon $ для метода сопряжённых градиентов равно $ 10^{-8} $.

\subsubsection*{Тест №1}

\[
 \varphi_2 = 2.0
\]

\[
 \varphi_3 = 3.0
\]

\[
 \varphi_4 = 1.0
\]

\[
  f = 0
\]

Имеем лишь ошибку округления.

\begin{table}[H]
\begin{center}
  \begin{tabular}{*{6}c}
    \toprule
    $ N_r $ & $ N_z $ & n & $ \delta_1 $ & $ \delta_2 $ & $ \delta_3 $ \\
    \midrule
    2 & 2 & 8 & $9.899e-16 $ & $3.92e-16 $ & $3.67e-16 $ \\
    4 & 4 & 23 & $8.69e-11 $ & $7.79e-11 $ & $5.59e-11 $ \\
    8 & 8 & 63 & $8.44e-10 $ & $4.12e-10 $ & $1.51e-10 $ \\
    16 & 16 & 126 & $3.92e-10 $ & $2.99e-10 $ & $7.03e-10 $ \\
    32 & 32 & 310 & $6.49e-09 $ & $7.98e-09 $ & $3.19e-09 $ \\
    \bottomrule
  \end{tabular}
  \caption{Явный метод}
\end{center}
\end{table}

\begin{table}[H]
  \begin{center}
    \begin{tabular}{*{6}c}
      \toprule
      $ N_r $ & $ N_z $ & n & $ \delta_1 $ & $ \delta_2 $ & $ \delta_3 $ \\
      \midrule
      2 & 2 & 5 & $6.52e-16 $ & $7.22e-16 $ & $3.38e-16 $ \\
      4 & 4 & 8 & $2.789e-11 $ & $4.15e-11 $ & $9.47e-11 $ \\
      8 & 8 & 16 & $2.805e-10 $ & $9.75e-10 $ & $3.06e-10 $ \\
      16 & 16 & 23 & $9.96e-10$ & $1.96e-10 $ & $6.67e-10 $ \\
      32 & 32 & 49 & $2.97e-9 $ & $9.76e-9 $ & $3.46e-9 $ \\
      \bottomrule
    \end{tabular}
    \caption{Неявный метод}
  \end{center}
\end{table}

\subsubsection*{Тест №2}

\[
 \varphi_2 = 2.0
\]

\[
 \varphi_3 = 3.0
\]

\[
 \varphi_4 = 1.0
\]

\[
  f = 0
\]

Присутствует лишь ошибка округления.

\begin{table}[H]
  \begin{center}
    \begin{tabular}{*{6}c}
      \toprule
      $ N_r $ & $ N_z $ & n & $ \delta_1 $ & $ \delta_2 $ & $ \delta_3 $ \\
      \midrule
      2 & 2 & 8 & $5.31e-16 $ & $4.38e-16 $ & $2.10e-16 $ \\
      4 & 4 & 25 & $1.86e-11 $ & $7.96e-11 $ & $9.29e-11 $ \\
      8 & 8 & 60 & $2.71e-10 $ & $9.19e-10 $ & $1.81e-10 $ \\
      16 & 16 & 111 & $6.7e-10 $ & $6.74e-10 $ & $1.41e-10 $ \\
      32 & 32 & 285 & $5.71e-9 $ & $8.04e-9 $ & $9.09e-9 $ \\
      \bottomrule
    \end{tabular}
    \caption{Явный метод}
  \end{center}
  \end{table}
  
  \begin{table}[H]
    \begin{center}
      \begin{tabular}{*{6}c}
        \toprule
        $ N_r $ & $ N_z $ & n & $ \delta_1 $ & $ \delta_2 $ & $ \delta_3 $ \\
        \midrule
        2 & 2 & 5 & $3.57e-16 $ & $5.312e-16 $ & $3.31e-16 $  \\
        4 & 4 & 8 & $5.51e-11 $ & $9.49e-11 $ & $6.03e-11 $ \\
        8 & 8 & 16 & $2.88e-10 $ & $1.67e-10 $ & $9.416e-10 $ \\
        16 & 16 & 24 & $5.48e-10 $ & $5.97e-10 $ & $9.06e-10 $ \\
        32 & 32 & 41 & $1.77e-9 $ & $2.28e-9 $ & $1.61e-9 $ \\
        \bottomrule
      \end{tabular}
      \caption{Неявный метод}
    \end{center}
  \end{table}

\subsubsection*{Тест №3}

\[
 \varphi_2 = 2(z + 1) + 2(1 + 2z)^2(z + 1)
\]

\[
 \varphi_3 = -9r^4 + 3r^2
\]

\[
 \varphi_4 = 3r^2
\]

\[
  f = - 8 r^{2}\left(3 r + 4 z\right) - 4 r \left(r + 2 z\right) \left(z + 1\right) - 4 \left(r + 2 z\right)^{2} \left(z + 1\right)
\]

Присутствует ошибка округления и ошибка аппроксимации.

\begin{table}[H]
  \begin{center}
    \begin{tabular}{*{6}c}
      \toprule
      $ N_r $ & $ N_z $ & n & $ \delta_1 $ & $ \delta_2 $ & $ \delta_3 $ \\
      \midrule
      2 & 2 & 7 & $7.882e-2 $ & $3.49e-2 $ & $2.192e-2 $ \\
      4 & 4 & 21 & $1.962e-2 $ & $8.761e-3 $ & $5.477e-3 $ \\
      8 & 8 & 53 & $4.948e-3 $ & $2.186e-3 $ & $1.358e-3 $ \\
      16 & 16 & 142 & $1.222e-3 $ & $5.39e-4 $ & $3.419e-4 $ \\
      32 & 32 & 247 & $3.377e-4 $ & $1.266e-4 $ & $8.486e-5 $ \\
      \bottomrule
    \end{tabular}
    \caption{Явный метод}
  \end{center}
\end{table}
  
\begin{table}[H]
  \begin{center}
    \begin{tabular}{*{6}c}
      \toprule
      $ N_r $ & $ N_z $ & n & $ \delta_1 $ & $ \delta_2 $ & $ \delta_3 $ \\
      \midrule
      2 & 2 & 5 & $7.882e-2 $ & $3.49e-2 $ & $2.192e-2 $ \\
      4 & 4 & 9 & $1.962e-2 $ & $8.761e-3 $ & $5.477e-3 $ \\
      8 & 8 & 15 & $4.948e-3 $ & $2.186e-3 $ & $1.358e-3 $ \\
      16 & 16 & 26& $1.222e-3 $ & $5.39e-4 $ & $3.419e-4 $ \\
      32 & 32 & 45 & $3.377e-4 $ & $1.266e-4 $ & $8.486e-5 $ \\
      \bottomrule
    \end{tabular}
    \caption{Неявный метод}
  \end{center}
\end{table}

\subsubsection*{Тест №4}

\[
  \varphi_2 = 2(z^2 + z + 1) + 2(1 + 2z)^3 (z^2 + z + 1)
\]

\[
 \varphi_3 =  -27r^5 + 3r^2
\]

\[
 \varphi_4 =7r^2
\]

\[
  f = - 2 r^{2} \left(3 r + 4 z\right)^{2} \left(3 r + 16 z + 6\right) - 6 r \left(r + 2 z\right)^{2} \left(z^{2} + z + 1\right) - 4 \left(r + 2 z\right)^{3} \left(z^{2} + z + 1\right)
\]

Присутствует ошибка округления и ошибка аппроксимации.

\begin{table}[H]
  \begin{center}
    \begin{tabular}{*{6}c}
      \toprule
      $ N_r $ & $ N_z $ & n & $ \delta_1 $ & $ \delta_2 $ & $ \delta_3 $ \\
      \midrule
      2 & 2 & 9 & $5.68e-2 $ & $6.32e-2 $ & $7.45e-2 $ \\
      4 & 4 & 27 & $1.422e-2 $ & $1.579e-2 $ & $1.854e-2 $ \\
      8 & 8 & 54 & $3.56e-3 $ & $3.937e-3 $ & $4.67e-3 $ \\
      16 & 16 & 138 & $8.87e-4 $ & $9.815e-4 $ & $1.161e-3 $ \\
      32 & 32 & 284 & $2.108e-4 $ & $2.38e-4 $ & $2.846e-4 $ \\
      \bottomrule
    \end{tabular}
    \caption{Явный метод}
  \end{center}
\end{table}
  
\begin{table}[H]
  \begin{center}
    \begin{tabular}{*{6}c}
      \toprule
      $ N_r $ & $ N_z $ & n & $ \delta_1 $ & $ \delta_2 $ & $ \delta_3 $ \\
      \midrule
      2 & 2 & 5 & $5.68e-2 $ & $6.32e-2 $ & $7.45e-2 $ \\
      4 & 4 & 8 & $1.422e-2 $ & $1.579e-2 $ & $1.854e-2 $ \\
      8 & 8 & 15 & $3.56e-3 $ & $3.937e-3 $ & $4.67e-3 $ \\
      16 & 16 & 23 & $8.87e-4 $ & $9.815e-4 $ & $1.161e-3 $ \\
      32 & 32 & 42 & $2.108e-4 $ & $2.38e-4 $ & $2.846e-4 $ \\
      \bottomrule
    \end{tabular}
    \caption{Неявный метод}
  \end{center}
\end{table}

По значениям наших тестов можно сделать следующие выводы:
\begin{itemize}
  \item Неявный метод имеет значительно большую скорость сходимости.
  \item Увеличение числа разбиений в два раза приводит к уменьшению погрешности примерно в четыре раза при наличии ошибки аппроксимации.
\end{itemize}