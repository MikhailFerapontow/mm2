\section{Метод сопряжённых градиентов}

Для решения системы линейных алгебраических уравнений использовать метод
сопряженных градиентов с предобусловливанием. Опишем метод решения следующей системы:

\[
  Ax = b
\]

где A - симметричная, положительно определённая, матрица.

\subsection{Явный метод}

\RestyleAlgo{tworuled}
\SetAlFnt{\normalsize}
\SetAlgoNoLine

\begin{algorithm*}[H]
  \DontPrintSemicolon
  $r^{(0)} = b - Ax^{(0)} $ \;
  $s^{(1)} = r^{(0)} $ \;
  $ \gamma = \sqrt{(b, b)} $

  \For{$k = 1$ \KwTo $k_{max}$}{
    $\displaystyle
        \alpha_k = \frac{\left( r^{(k-1)},\ r^{(k-1)} \right)}{\left( As^{(k)},\ s^{(k)} \right)}
    $\;
    $x^{(k)} = x^{(k-1)} + \alpha_k s^{(k)}$\;
    $r^{(k)} = r^{(k-1)} - \alpha_k s^{(k-1)}$\;

    \If{$\sqrt{\left( r^{(k)},\ r^{(k)} \right)} < \gamma \varepsilon $}{
        \textbf{break}
    }

    $\displaystyle
        \beta_k = \frac{\left( r^{(k)},\ r^{(k)} \right)}{\left( r^{(k-1)},\ r^{(k-1)} \right)}
    $\;
    $s^{(k+1)} = r^{(k)} + \beta_k s^{(k)}$\;
}
\end{algorithm*}

\subsection{Неявный метод}

Неявный метод основывается на использовании предобуславливания. 
Идея заключается в том, чтобы выбрать матрицу \textbf{B}, которая является
симметричной и положительно определенной, и приблизительно равна 
матрице \textbf{A}. Мы выбираем \textbf{B} в виде
$B = \Tilde{L} \Tilde{L}^T $, где $\Tilde{L} \Tilde{L}^T$ представляет собой 
неполное разложение Холецкого для матрицы \textbf{A}.
Позже мы объясним, почему такой выбор матрицы \textbf{B} считается оптимальным.

\begin{algorithm*}[H]
  
  \DontPrintSemicolon

  $r^{(0)} = b - A x^{(0)}$\;
  $B w^{(0)} = r^{(0)}$\;
  $s^{(1)} = w^{(0)}$\;
  $B g = b$\;
  $\gamma = \sqrt{\left( g,\ b \right)}$\;

  \For{$k = 1$ \KwTo $k_{max}$}{
      $\displaystyle
          \alpha_k = \frac{\left( w^{(k-1)},\ r^{(k-1)} \right)}{\left( As^{(k)},\ s^{(k)} \right)}
      $\;
      $x^{(k)} = x^{(k-1)} + \alpha_k s^{(k)}$\;
      $r^{(k)} = r^{(k-1)} - \alpha_k s^{(k-1)}$\;
      $Bw^{(k)} = r^{(k)}$\;

      \If{$\sqrt{\left( w^{(k)},\ r^{(k)} \right)} < \gamma \varepsilon $}{
          \textbf{break}
      }

      $\displaystyle
          \beta_k = \frac{\left( w^{(k)},\ r^{(k)} \right)}{\left( w^{(k-1)},\ r^{(k-1)} \right)}
      $\;
      $s^{(k+1)} = w^{(k)} + \beta_k s^{(k)}$\;
  }
\end{algorithm*}

Для решения системы $Bw(k) = r(k)$ мы используем метод Гаусса.
При выборе $B = \Tilde{L} \Tilde{L}^T $, нам остается выполнить только две обратные подстановки:
$ \Tilde{L} z^{(k)} = r^{k} $ и $ \Tilde{L}^T w^{(k)} = z^{(k)}$.